% The Abstract Page
\addtotoc{アブストラクト}  % Add the "Abstract" page entry to the Contents
\abstract{
\addtocontents{toc}{\vspace{1em}}  % Add a gap in the Contents, for aesthetics

本研究では、日本語を第二言語とする学習者(L2)の習熟度をモデル化するために、言語的複雑性の指標とCriterial Features(基準特徴)の活用を検討する。IーJASコーパスの学習者作文を用い、これらの特徴量がL2の発達過程をどの程度反映し、かつ習熟度分類において解釈可能な予測を支援できるか分析した。

予測モデルでは、解釈性の高い機械学習手法ではExplainable Boosting Machine(EBM)を用い、特徴量と習熟度レベルの関係可視化した。5段階の習熟度分類において加重F1スコアは.38を達成し、ベースラインを上回った。

特徴量重要度分析の結果、主に言語的複雑性の指標において有効であることが明らかとなった。たとえば、文あたりの並列節数、Morphological Complexity Index(MCI)、Lexical Frequency
Profile(LFP)が予測において有効であることが明らかとなった。一方、基準特徴として設定された受身形も上位15個の重要特徴量として抽出された。

これらの知見は、習熟度の上昇にともなって言語的複雑性が増加するという第二言語習得理論と整合的である。一方で、従属節の早期出現や、上級段階における並列構造の継続的な増加といった予想外の傾向も観察され、教授順序や解析器の限界が影響している可能性が示唆された

    %ToDO finish writing Japanese trnaslation

Beyond modeling, the study also highlights developmental trajectories in complexity measures and discusses implications for language assessment, curriculum design, and learner feedback. Limitations include the use of uncorrected learner data, reliance on surface-form features, and imbalanced class sizes, especially for N1 learners. Future work can explore error-based features, discourse-level complexity, and longitudinal data collection. This research contributes to both second language acquisition theory and the development of computational tools for assessing L2 Japanese proficiency.
}

\clearpage 