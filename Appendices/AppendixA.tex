\chapter{Appendix}
\appendix

list of measures (I will make this pretty later)
Syntactic Complexity
\begin{itemize}
    \item Sentence Length
    \item Clause Length
    \item Clauses per sentence
    \item Coordinate Clauses per sentence
    \item Coordinate Clauses ratio to all clauses
    \item Subordinate Clauses per sentence
    \item Subordinate Clause Ratio to all clauses
    \item Subordinate Clauses to Coordinating Clauses
    \item Average Noun Phrase length
    \item Average Verb Phrase Length
\end{itemize}

Lexical Complexity
\begin{itemize}
    \item Corrected Type Token Ratio
    \item Average word length
    \item Noun Density
    \item Verb Density (including auxilaries)
    \item Adjective Density
    \item Adverb Density
    \item MTLD
    \item Lexical Frequency Profile
\end{itemize}

Morphological Complexity
\begin{itemize}
    \item MCI - 5
    \item MCI - 10
    \item KOMORA (MATTR and MTLD)
\end{itemize}

Criterial Features
\begin{longtable}{p{2cm} p{4cm} p{8cm}}
\caption{Grammar Forms extracted as criterial features from the learner texts.}
\label{tab:Criterial-Features}\\
\toprule
\textbf{JLPT Level} & \textbf{Grammar Form} & \textbf{English Equivalent}  \\
\midrule
\endfirsthead

\toprule
\textbf{JLPT Level} & \textbf{Grammar Form} & \textbf{English Equivalent}
\midrule
\endhead

\midrule \multicolumn{3}{r}{{Continued on next page}}\\
\midrule
\endfoot

\bottomrule
\endlastfoot

N1  &	あえて &	        dare to \\
N1  &	案の定 &	        just as one thought\\
N1  &	あらかじめ   &	beforehand; in advance\\
N1  &	ばこそ &	        only because\\
N1  &	どうにも○○ない    &	not … by any means\\
N1  &	ほうがましだ  &	I would rather\\
N1  &	いかなる	&       any kind of\\
N1	&   可能性がある  &	there's a possibility\\
N1	&   かつて	&       once; before\\
N1	&   嫌いがある &     	to have a tendency to\\
N1	&   きりがない	&   there's no end to\\
N1	&   もしくは	    &   or; otherwise\\
N1	&   てみせる	    &   I'll definitely\\
N1	&   てしかるべきだ & 	should\\
N1	&   という     &   	all; every\\
N1	&   とは	    &       (indicates word or phrase being defined); or expression of surprisal\\
N1  &	とはいえ	&       nonetheless; although\\
N4	&   以上  	&       over\\
N2	&   以上に	&       more than; no less than\\
N2	&   えない    &     	unable to; cannot\\
N2	&   える / うる &	can; is possible\\
N2	&   お○○願う &     	could you please\\
N2	&   恐れがある &     	there are fears that\\
N2	&   かいがある &     	it's worth one's effort to do something\\
N2	&   限り	        &   as long as; while… is the case\\
N2	&   かと思ったら / かと思うと	& then again; just when; no sooner than\\
N2	&   から言うと   &	in terms of; from the point of view of\\
N2	&   からして	&   judging from; based on\\
N2	&   からすると / からすれば	&   judging from; considering\\
N3	&   さらに    &   	furthermore; again; more and more\\
N2	&   しかも	&       moreover; furthermore\\
N2	&   次第	     &      as soon as\\
N2	&   次第だ / 次第で &	depending on; so\\
N2	&   その上	&       besides; in addition; furthermore\\
N3	&   それとも  &	    or; or else\\
N2	&   それにしても & 	nevertheless; even so\\
N3	&   だけでなく   &	not only… but also\\
N2	&   つつ	&           while; although\\
N3	&   つもりで	&       with the intention of doing\\
N2	&   でしかない	&   merely; nothing but; no more than\\
N3	&   てしょうがない & 	very; extremely\\
N2	&   てたまらない	&   very; extremely; can't help but do\\
N2	&   て当然だ	&       natural; as a matter of course\\
N2	&   てはいられない & 	can't afford to; unable to\\
N2	&   というものだ  &	something like…; something called…\\
N2	&   と考えられる &	one can think that…\\
N3	&   ないことにはない &	states something is not quite impossible but requires great effort\\
N2	&   ないではいられない & can't help but feel; can't help but do\\
N2	&   なお	    &        furthermore; still; yet (used to add more information to the )\\
N2	&   にかかわらず &	regardless of\\
N2	&   に限って &	    only; particularly when\\
N2	&   に限らず &	    not just; not only… but also\\
N2	&   に限る &       	nothing better than; there's nothing like\\
N2	&   に決まっている &	I'm sure that…\\
N2	&   に応えて	    &   in response to\\
N2	&   に基づいて   &   	based on\\
N2	&   に過ぎない	&   no more than; just; merely\\
N2	&   にもかかわらず &	despite; in spite of; although\\
N2	&   ねばならない  &	have to; must\\
N2	&   果たして    &  	sure enough; really\\
N2	&   ふうに	&       in a way (this way/that way/what way)\\
N3	&   ぶりに	&       for the first time in\\
N3	&   むしろ	&       rather; instead\\
N2	&   もかまわず   &	without worrying about\\
N2	&   もっとも    &	but then; although\\
N2	&   ものがある &     	(sentence-ending expression of strong judgement)\\
N2	&   ものだから   &	because; the reason is\\
N2	&   ものの      &   	but; although\\
N2	&   やがて &       	soon; before long; eventually\\
N2	&   やら○○やら  &	such things as\\
N2	&   よりほかない  &	to have no choice but\\
N3	&   あまり     &   	so much… that \\
N3	&   あまりに    &	so much… that; too…\\
N3	&   いくら○○ても / いくら○○でも   &	no matter how\\
N3	&   一方だ	    &   more and more; continue to \\
N2	&   一方で	&   on one hand; on the other hand \\
N3	&   うちに &   	while; before; doing \\
N3	&   おかげで &	thanks to; because of \\
N3	&   がたい &   	hard to; difficult to\\
N3	&   ぎみ  &   	-like; -looking\\
N3	&   きる  &	    to do something completely\\
N3	&   切れない	 &  being too much to finish\\
N3	&   くせに &   	even though; and yet; despite\\
N3	&   決して○○ない &	never; by no means \\
N3	&   こそ	    &   certainly (emphasises the previous word)\\
N4	&   ことがある   &	can; sometimes happens\\
N3	&   ことはない   &	there is no need to; never happens\\
N3	&   さえ	    &   even  \\
N3	&   しかない    &	have no choice but\\
N3	&   ずに  &   	without (doing) \\
N3	&   せいで &   	because of  \\
N3	&   だけど &   	but; however\\
N3	&   確かに &	    surely; certainly\\
N3	&   たびに &   	everytime; whenever\\
N3	&   ついでに   &	taking the opportunity; while (you) are at it\\
N3	&   っぱなし  &	    leaving something still in use\\
N3	&   っぽい &   	    -ish; -like \\
N3	&   つまり &   	    in other words; that is to say\\
N4	&   という &   	    called; named\\
N3	&   というより  &    rather than\\
N3	&   と共に &   	    together with\\
N3	&   とは限らない &	not necessarily so; is not always true\\
N3	&   ながら / いながら / ながらも	&   although; despite \\
N3	&   なぜなら / なぜかというと  &	because; the reason is\\
N3	&   なるべく    &	as much as possible\\
N3	&   に比べて	&       compared to\\
N3	&   に違いない   &	I'm sure; no doubt that\\
N3	&   ばよかった   &	should have; it would have been better if\\
N3	&   べき       &   	must do; should do\\
N3	&   ほど	    &       the more; to the extent that; so much… that\\
N3	&   もしかしたら  &	perhaps; maybe\\
N3	&   ような気がする &	have a feeling that; think that\\
N4	&   受身形 &        Passive Form\\
N4	&   あまり○○ない &	not very; not much\\
N4	&   かしら &       	I wonder\\
N4	&   がする &       	smell; hear; taste\\
N4	&   かもしれない  &	might; maybe\\
N4	&   ございます   &	there is (honorific ある)\\
N4	&   させられる   &	to be made to do something\\
N4	&   させる &       	to make/let somebody do something\\
N4	&   しか○○ない &	    only; nothing but\\
N5	&   すぎる     &   	too much\\
N4	&   たところ    &   	just finished doing; was just doing\\
N4	&   たばかり    &   	just did; something just happened\\
N4	&   たら  &       	if; after; when\\
N4	&   たらどう    &	why don't you\\
N5	&   たり○○たり  &	do such things like\\
N4	&   ていただけませんか&	could you please\\
N4	&   ているところ  &	in the process of doing\\
N4	&   てくれる    &	to do something for someone \\
N4	&   てしまう / ちゃう  &	to do something (regretfully); to do something (completely) \\
N4	&   てすみません  &	I'm sorry for\\
N4	&   てもらう    &	to get somebody to do something\\
N4	&   と思う &       	I think; you think\\
N4	&   ところ     &   	about to; on the verge of\\
N5	&   なくてはいけない / なくてはならない &	must do; have to do\\
N4	&   なさい	&       order somebody to do something\\
N4	&   なさる &       	to do (honorific する) \\
N4	&   にくい     &   	difficult to\\
N4	&   のように / のような &	like; similar to\\
N4	&   みたい &       	like; similar to; resembling\\
N4	&   みたいに/みたいな   &	like; similar to\\
N4	&   やすい &       	easy to; likely to\\
N4	&   ようだ &       	it seems that; it appears that; it looks like\\
N4  &	より  &       	than\\
N4	&   られる1    &   	to be able to do something\\
N4	&   られる2    &   	to do (by someone) (passive 受身形)\\
N5	&   いちばん    &	the most \\
N3	&   くらい / ぐらい   &	about; approximately\\
N5	&   たい  &	want to\\
N5	&   つもり &	plan to; intend to\\
N5	&   てください   &	please do…\\
N5	&   ほうがいい   &	it'd be better to, state a preference\\
N5	&   ほうがいい2  &   	it'd be better to not\\

\end{longtable}

