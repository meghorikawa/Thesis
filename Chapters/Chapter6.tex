\chapter{Conclusion and Outlook}
% Where the research questions answered? - % summarize findings briefly again.


%2. To examine how linguistic complexity features and criterial features develop across different proficiencylevels
%in Japanese learners.
Aligning with previous studies as proficiency increases learners use more "parts" in their writings so sentence
length and clause length increases and diversity of vocabulary also increases. Mean Hierarchtical distance and mean
dependcy distance
also increases meaning
that more complex embedding is done in sentences.

%3. To assess whether linguistic complexity features and/or Criterial Features can reliably predict proficiency
%levels in L2 Japaneselearners.

Feature important analysis revealed that syntactic complexity (coordinate clause measures) and lexical
sophistication (JLPT vocabulary percentages) were the primary drivers of classification. This alsigns with findings
in prior linguistic complexity studies which show that as learners progress, their linguistic output becomes more
sophisticated in terms of sentence structure and vocabulary. However, despite the clear importance of these features,
the overall classification accuracy suggests that the current set of features, may not fully capture the
fine-grained distinctions required for highly accurate multi-level classification. Especially as learner errors
which have also been investigated alongside as a domain of Criterial features have not been implemented in this study.

%1. To identify which linguistic complexity measures, and criterial features show reliability as indices of language
%development in L2 Japanese learners.




% additional research areas/exploration.
EGP equivalent for Japanese additional validation of forms used.

Ideas: explore-accommodation for the third domain - Fluency (possibly using timing??)  Focus on Accuracy together...

More accurate analysis of clauses needed to answer additional questions about development of clausal complexity

This tool could also be useful for approximating the JLPT level a learner should take.

better analysis of てform? i.e. coordinate clauses, it is included in the grammar forms connected with other parts
but not on its own. The use of てform most likely makes up a good percentage of the coordinate clauses. (even through
the て is marked as SCONJ)


*