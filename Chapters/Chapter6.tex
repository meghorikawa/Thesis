\chapter{Conclusion and Outlook}
% Where the research questions answered? - % summarize findings briefly again.
This study set out to investigate which linguistic complexity measures and critierial features best reflect
developmental changes across L2 Japanese proficiency levels, and whether such features could be used to build an
interpretable, data-drive classification model.

%1. To identify which linguistic complexity measures, and criterial features show reliability as indices of language
%development in L2 Japanese learners.


%3. To assess whether linguistic complexity features and/or Criterial Features can reliably predict proficiency
%levels in L2 Japaneselearners.
The EBM classifier achieved an overall accuracy of 38\% outperforming the 20\% for five-class classification. While
this reflects only moderate predictive power, it underscores the inherent challenge of fine-grained proficiency
classification, especially between adjacent JLPT leves. Notably, performance improved substantially when levels were
grouped into broader categories, with an F1 score of .54. This result points to the continuum-like nature of the
JLPT scale, suggesting that its proficiency bads are not as discrete in practice as their labels imply. The small
number of N1 texts also limited classificaiton accuracy at the advanced level.

The most predictive features spanned multiple domains of linguistic complexity. Key indicators included syntactic
elaboration (e.g. coordinate clauses per sentence), morphological diversity (e.g. MCI, JRMA-MATTR), and lexical
sophistication (JLPT vocabulary list, and Out of Vocabulary List). These results are consistent with theoretical
expectations that more advanced learners produce longer, more complex sentences with greater lexical variation and
denser morphology. The prominence of coordinate clauses, in particular, what somewhat unexpected but aligns with a
broader pattern of structural elaboration at higher levels.

%2. To examine how linguistic complexity features and criterial features develop across different proficiencylevels
%in Japanese learners.
The evaluation of individual features confirmed many of the trends observed in the model. Measures such as sentence
length, clauses per sentence, coordinate clause use, mean dependency distance (MDD), mean hierarchical distance (
MHD), and mid- to upper-level vocabulary proportions (i.e. N4/N2) all showed consistent growth with proficiency and
clear discriminatory potential. Conversely, some features such as subordinate clauses, noun/verb density, and
type-token ratios, displayed plateau effects or less reliable patterns, especially beyond the intermediate levels.
These inconsistencies may reflect instruction-based usage patters or limitations in tokenization and clause
segmentation.

Unexpected findings, such as early use of subordination and sustained growth in coordination, point to the influence
of instructional sequencing or writing tasks. Additionally, the treatment of the て-form in parsing may have led to
underestimation of its contribution to syntactic complexity.

\section{Theoretical and Pedagogical Implications}
These results reinforce the view that L2 proficiency is a multi-dimential construct, not easily captured by a single
metric. The moderate classification performance across five levels, coupled with significant improvement using
three-group clustering, suggests that proficiency development in Japanese (similar to other languages) is gradual and
continuous. Findings on coordination diverge from common L2 developmental sequences that prioritize subordination,
suggesting language-specific or instructional influences in Japanese acquisition. Together, the results contribute
to a more nuanced understanding of syntactic, morphological, and lexical development in Japanese learners.

From an assessment perspective, the findings suggest that while certain features reliably indicate proficiency
growth, especially across broad stages, distinguishing fine-grained adjacent levels may be more difficult. This has
implications for both automated scoring and curriculum design. For example, emphasizing a diverse range of
grammatical constructions, including coordination, may better reflect real learner development. These features could
also be incorporated into automated feedback tools, although care must be
taken given the variability in learner
output and the limitations of current parsing tools for Japanese.

\section{Limitations}

Several limitations impacted the findings of this study. First, the learner corpus used (I-JAS) contained uncorrect
texts, with spelling and grammatical errors that likely affected NLP-based feature extraction, especially in
tokenization and POS tagging. Second, Japanese-specific morphological structures-such as compound verbs and
contextualized conjunctions posed challenges for rule-based extraction, potentially underestimating raw frequency
measures. 

This study soley focused on surface-form and structural measures. While these are foundational for complexity
analysis, they do not account for use-based, semantic, pragmatic, or discourse-level features, which may offer
additional insight. The JLPT scores, inferred via the J-CAT placement test, were used as the proxy for proficiency,
but may not perfectly reflect productive language ability. Finally, the small number of N1 texts limited the model's
capacity to accurately capture the most advanced level.

\section{Future Work}
% additional research areas/exploration.
EGP equivalent for Japanese additional validation of forms used.

Ideas: explore-accommodation for the third domain - Fluency (possibly using timing??)  Focus on Accuracy together...

More accurate analysis of clauses needed to answer additional questions about development of clausal complexity

This tool could also be useful for approximating the JLPT level a learner should take.

better analysis of てform? i.e. coordinate clauses, it is included in the grammar forms connected with other parts
but not on its own. The use of てform most likely makes up a good percentage of the coordinate clauses. (even through
the て is marked as SCONJ)

\section{Concluding Remarks}

