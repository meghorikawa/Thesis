\chapter{Methodology}
% information on the corpus and writing data
%define  features that I will use for classification how I decided on this and motivation behind these features.
%%%%%%%%%%%%%%%%%%%%%%%%%%%%%%%%%%%%%%%%
\section{Characteristics of Japanese}
%%%%%%%%%%%%%%%%%%%%%%%%%%%%%%%%%%%%%%%%

The agglutinative nature of Japanese coupled with the absence of explicit word delimiters, presents challenges for
accurate word segmentation. Furthermore, the use of three distinct writing systems
\footnote{Hiragana, Katakana, and Kanji} leads to considerable orthographic variation. for instance, learners at
lower proficiency levels frequently rely on the phonetic alphabets (hiragana and katakana), which can exacerbate
segmentation and tokenization errors. This phenomenon has been observed in previous studies
\citep{yang1998, nagata2009}, where systems demonstrated a lack of robustness against "spelling" errors or the
erroneous use of kanji.

Defining a "word" in Japanese also differs considerably from Indo-European languages. While a bound morpheme in
Enlgish might be treated as a indificual word. this is generally not the case in Japanese. For example,
segmentation can be difficult.
With 3
alphabets used there can orthographic variation is also fairly common. Learners at the lower proficentcy levels
mostly will write using the phonetic alphabets hiragana and katakana sometimes leading to errors in
segmentation/tokenization as was observed in \citep{yang1998, nagata2009}.

What is considered a word in Japanese may differ from other languages european languages. While a bound morpheme in
English might be treated
as an individual word, this is generally not the case in Japanese. For example, the word 話す\textit{hanasu}(to speak)
is considered a single word. However its potential form, 話せる \textit{hanaseru}(to be able to speak) is often analyzed as
two distinct morphemes: 話\textit{hana} and せる\textit{seru}. Initial abalyses using the chose tokenizer and parser
revealed inconsistencies in this regard. Similar inconsistencies were observed with compound verbs, such as
言い切る(\textit{iikiru}, "to completely say\footnote{\textit{to say without restraint}}"). While this was consistently
split into 言い(\textit{ii}"to say") and 切る(\textit{kiru} a suffix indicating completion), other instances were
segmented into
食べ (\textit{tabe}"to eat") 切る(\textit{kiru}"completely"). These inconsistencies pose significant hurdles for the
reliable extraction of specific grammatical forms for criterial features and can lead to unreliable counts for
complexity measures.
% maybe say something about taking this into consideration when writing rules for criterial features.

%%%%%%%%%%%%%%%%%%%%%%%%%%%%%%%%%%%%%%%%
\section{About the International Corpus of Japanese as a Second Language(I-JAS)}
%%%%%%%%%%%%%%%%%%%%%%%%%%%%%%%%%%%%%%%%

This study utilizes data from the International Corpus of Japanese as a Second Language (I-JAS), as
detailed in
\citet{Sakoda2020}, was
used.  The
I-JAS corpus is comprised of both spoken and written samples of Japanese.  It includes data from a diverse pool of 1,
000 adult learners
(aged
between 17 and 63 years old), all of whom are learning Japanese as a second language. 50 Native speaker samples are also
included in the corpus as a control group.

Participant's proficiency levels were assessed using the Japanese Computer Adaptive Test (J-Cat)
\citep{Imai2009}, with Further details about this assessment is provided in the
subsequent
section \ref{j-cat}. In addition to proficiency scores, the corpus includes various metadata for each participant,
such as
their
native language,
prior experience of visiting or living in Japan, and current geographical location (whether outside or within Japan).

%%%%%%% This part below contains information only on the essay writing samples I previously analyzed. I have also
%%%%%%% included additional writing samples added to the corpus which has brought the total particiapnt pool to 1000
%%%%%%% again.
Writing samples were extracted from the larger I-JAS corpus. Samples  were provided from 687 individuals, including the
control group
of 50 native speakers. A detailed breakdown of the of the participants in the writing sample subset, categorized by
their corresponding Japanese Language Proficiency Test (JLPT) levels, is presented in \ref{tab:participants-chart}.

%Name: count, dtype: int64
\begin{table}[h!]
\centering
\begin{tabular}{cc}
\hline \textbf{JLPT Proficiency Level} & \textbf{\# of Participants} \\ \hline
N5 & 176 \\
N4  & 318 \\
N3 & 297\\
N2 & 165 \\
N1 & 44 \\
Native Speakers & 50 \\
\hline
\end{tabular}
\caption{Distribution of participants across JLPT proficiency levels. J-cat scores have been mapped to their equivallent JLPT levels. }
\label{tab:participants-chart}
\end{table}

%%%%%%%%%%%%%%%%%%%%%%%%%%%%%%%%%%%%%%%%
\subsection{The Japanese Computerized Adaptive Test (J-CAT)}
\label{j-cat}
%%%%%%%%%%%%%%%%%%%%%%%%%%%%%%%%%%%%%%%%

%background and information on the J-CAT test compare to JLPT.

The Japanese Computerized Adaptive Test (J-CAT) \citep{Imai2009}, is a computer-administered assessment
designed to evaluate an
individual's proficiency in the Japanese language. While formerly freely accessible, the J-CAT is now overseen by
the  日本語教育支援協会(Japanese Language Education Support Association (JaLESA)). Japanese
universities frequently
employ the J-CAT as an efficient tool and flexible tool for placing foreign students into appropriate Japanese
language courses, primarily due to its on-demand administration compared to the JLPT which is only
administered bi-annually.

The adaptive nature of the J-CAT allows it to tailor question difficulty based on a student's performance across
four core areas: Vocabulary, Grammar, Listening, and Reading. Each participant receives a numberial score, which is
then mapped to one of seven distinct proficiency levels. These levels have been correlated with equivalent JLPT
scores, as detailed in \ref{tab:proficency-table}. For this study participants were categorized according to their
assigned JLPT level. Native speaker participants were assigned a default J-CAT score of 999.

\begin{figure}[h!]
    \centering
    \includegraphics[scale=.3]{img/JCatScores.png}
    \caption{The assigned proficency levels from the J-Cat test as presented in the original paper in  2009. It is important to note that JLPT equivalencies do not align with the current JLPT framework, which was reformed in 2010. A revised score interpretation, connecting J-CAT scores to the updated JLPT levels, was released in 2011 and is provided in Table \ref{tab:proficency-table} }
    \label{fig:JCatLevels}
\end{figure}


\begin{table}[h!]
\centering
\begin{tabular}{lrl}
\hline \textbf{JLPT Proficiency Level} & \textbf{J-Cat Score}  \\ \hline
N5 & 0 - 149 \\
N4 & 150 - 199 \\
N3 & 200 - 249 \\
N2 & 250 - 299 \\
N1 & 300 - \\
Native & 999\\
\hline
\end{tabular}
\caption[Proficency Levels]{JLPT proficency level classification based on J-cat score ranges, adapted from
\cite{jcat_interpretation_guide}.}
\label{tab:proficency-table}
\end{table}

%%%%%%%%%%%%%%%%%%%%%%%%%%%%%%%%%%%%%%%%
\subsection{Writing Tasks}
%%%%%%%%%%%%%%%%%%%%%%%%%%%%%%%%%%%%%%%%

% add the additional SW 1 and 2 tasks where participants were expected to write a story based on a picture.
Each participant submitted up to six samples of writing for specific tasks, detailed below that were designed to
elicit a variety of linguistic responses across different discourse types and levels of formality:
\begin{itemize}
    \item (\textbf{Task e}), A short essay titled "Our Eating Habits," requiring a comparatie analysis of fast food
    and home-cooked
    meals within the context of the learner's home country.
    \item (\textbf{Task m1}), A formal letter addressed to a former teacher, requesting a letter of recommendation
    for a scholarship
    application.
    \item (\textbf{Task m2}), An email seeking an extension for a report submission deadline.
    \item An apology email (\textbf{m3}) declining an invitation to give a close friend a sight-seeing tour.
    \item (\textbf{Task SW1}), A story-telling task, where the learner was expected to narrate a story based on a
    series of pictures depicting a picnic theme. Sample images for this task are also shown in Figure \ref{fig:ST}
    in the appendix.
    \item (\textbf{Task SW2}), A story telling task where the learner was expected to narrate a story based on a
    series of pictures depicting a lost key. Sample images for this task can be seen in Figure \ref{fig:ST} in the
    appendix.
\end{itemize}

These tasks were standardized across all participants, regardless of their proficiency levels, encompassing a range
of communicative functions and formalities. Average text length can be see to vary widely across the task as shown
in figure \ref{fig:text-lengths}.
While the possibility of a "task
effect" as
observed in
\citet{Alexpoulou2017} is possible due to certain tasks requiring the use of certain forms, as the tasks are
consistent across proficiency levels,
this consistency allows for a direct observation of the linguistic forms learners use at different proficiency
levels across tasks. Consequently, all writing samples within the corpus were processed as-is, without any
correction of learner
errors.

\begin{figure}[h!]
    \centering
    \includegraphics[scale=.5]{img/text_length_chart}
    \caption{The average text length across task and JLPT level. A detailed chart is
    included in the Appendix table \ref{tab:text_len}}
    \label{fig:text-lengths}
\end{figure}

%%%%%%%%%%%%%%%%%%%%%%%%%%%%%%%%%%%%%%%%
\subsection{Text Preprocessing}
%%%%%%%%%%%%%%%%%%%%%%%%%%%%%%%%%%%%%%%%

%Mention the preprocessing of the text. Talk about spacys' ginza pipeline used for parsing-tagging etc.
The raw text data from each writing sample underwent a comprehensive preprocessing pipeline. Given the specific
challenges of Japanese text processing, particularly for learner language where standard NLP tools may not be
robust, Spacy's Ginza package \cite{Ginza} (v 5.2) a Japanese language model for spaCy for Japanese was employed.
Ginza was selected for its comprehensive capabilities in handling Japanese text, including tokenization,
Part-of-Speech(POS) tagging, depedency parsing, lemmatization, morphological analysis, and its integration with
spaCy's rule based matcher.

For certain morphological lexical and measures, punctuation was removed from the text to prevent its influence on
token counts and other analyses. All characters in the writing were first converted to full-size characters \footnote{This is to prevent any errors due to unintentional half-size characters that may have been included in the writings when converting between different formating. For this the mojimoji package was used https://pypi.org/project/mojimoji/}, then
the texts were tokenized, tagged for parts-of-speech,
Lemmatizatized, dependency parsed.

The corpus for this study comprises over 4,840 uncorrected samples of writing from the participants, necessitating
the reliance on measures that can be automatically calculated.
 The algorithms for calculating various complexity measures and criterial features, discussed in subsequent
sections, were
were developed by the author based on the output of this preprocessing pipeline. The source code for these
algorithms in publicly available at: \href{https://github.com/meghorikawa/JFE}{https://github.com/meghorikawa/JFE} .

%%%%%%%%%%%%%%%%%%%%%%%%%%%%%%%%%%%%%%%%
\section{Complexity Measures}
%%%%%%%%%%%%%%%%%%%%%%%%%%%%%%%%%%%%%%%%
To capture the multifaced nature of language development, this study uses a multi-dimensional approach to complexity
measurement. Measures were chosen across multiple linguistic domains- including lexical, syntactic and morphological
complexity - at both gloabl and local levels. The following sections describe the specific measures chosen for each
domain. A comprehensive list of all complexity measures extracted in this study is located in the Appendix in
Table \ref{tab:complexity-measures}

%%%%%%%%%%%%%%%%%%%%%%%%%%%%%%%%%%%%%%%%
\subsection{Syntactic Complexity Measures}
%%%%%%%%%%%%%%%%%%%%%%%%%%%%%%%%%%%%%%%%
Syntactic complexity is assessed using a combination of global and local measures, which together provide a fuller
picture of learners' syntactic development. Global measures reflect complexity at the sentence or clause level,
and illustrate a learner's ability to build extended structures. Local measures focus on phrase-level elaboration
and the richness of internal units.

Due to the large size of the corpus, one of the main criteria for choosing these measures was the ease with which
they could be automatically calculated. For this reason, T-unit based measures, which require more complex and often
manual annotation, were not included.

\subsubsection{Global Measures}
Among the global measures, average sentence length and clauses per sentence were included. Sentence length is
calculated as the average number of tokens in each sentence, while clauses per sentence counts both finite and
non-finite clauses divided by the number of sentences. These measures indicate how much structural elaboration
learners achieve, as longer sentences/clauses indicate more units used.

In addition, the frequency of subordinate and coordinate clauses per sentence was used to examine learners' use of
clausal embedding.
Subordinate clauses represent hierarchical syntactic relationships (e.g. conditionals or complements),
while coordinate clauses reflect additive or sequential connections between clauses. These measures provide insight
into learners'
use of complex sentence structures. A similar method was employed in \citet{Vyatkina2012}, where the frequency of
subordinating and
coordinating conjunctions as a proxy for coordination and subordination
Following Vyatkina's approach, this study also uses the frequency of surface forms as a proxy for subordination and
coordination. To
allow for comparability across text of different lengths, these counts were normalized by total word count using the
following formula:
% this is the formula
\begin{center}
${\displaystyle \frac{\# \hspace{5pt} of \hspace{5pt}SC \hspace{5pt}or \hspace{5pt}CC}{total \hspace{5pt} \hspace{
5pt}word \hspace{5pt}count} }  * 100$
\end{center}
where SC and CC refer to subordinating and coordinating conjunctions, respectively.

However, automatically identifying subordinate and coordinate clauses in Japanese in challenging. The POS tag for
subordinating conjunctions (SCONJ) is often applied to particles that may function as coordinators or subordinators
depending on the context. Also, dependency labels that mark coordination are not always reliably assigned by parsers
due to ambiguity in structure \citep{UDJapanese}. Because of this, standard automatic annotation may not clearly
differentiate coordination from subordination in Japanese.

To work around this, a rule-based approach was used to approximate the frequencies of subordinate and coordinate
clauses. This method relies on detecting specific surface forms of conjunctions such as 「から」(kara) and 「ので」(node)
for subordination, or chains of te-forms for coordination. Although this may not capture non-standard constructions,
it is a practical way to estimate clause types in a large corpus.

\subsubsection{Local Measures}
On the local level, measures such as noun phrase length, verb phrase length, mean dependency distance (MDD), and mean
hierarchical
distance (MHD) were included. Noun phrase length is the average number of tokens per noun phrse, reflecting phrase
complexity, or more descriptiveness by adding adjectives to compliment nouns.  Similar to noun phrase length, verb
phrase length also analyzes the length of verb phrases and any adverbs, complementizers, used.  MDD and MHD,
indicate linear and hierarchical syntactic complexity and offer a more detailed view of sentence processing demands.

%%%%%%%%%%%%%%%%%%%%%%%%%%%%%%%%%%%%%%%%
\subsection{Lexical Complexity Measures}
%%%%%%%%%%%%%%%%%%%%%%%%%%%%%%%%%%%%%%%%
%Corrected Type Token Ratio
%Noun Density
%Verb Density (including auxilaries)
%Adjective Density
%Adverb Density
%MTLD
%Lexical Frequency Profile

Lexical complexity in this study is analyzed using both text internal measures and
text
external measures. Text internal measures are based solely on the vocabulary used within each text and measure the
diversity and density of the vocabulary in a text. In contrast, text-external measures assess sophistication by
comparing the vocabulary in a learner text to an external source such as a
frequency list or graded vocabulary resources.

\subsubsection{Text Internal Measures}
%internal
The internal measures selected for this study include the
Corrected-Type-Token-Ratio(CTTR),Measure of Textual Lexical Diversity (MTLD), and part of speech (POS) density
measures, including noun density, verb density(including auxiliaries), adjective density and adverb density.

CTTR is a refinement of the traditional type-token ratio that reduces sensitivity to text length. It is calculated
using the formula:
%%%%CTTR Formula
\begin{center}
    \centering CTTR = ${\displaystyle \frac{total \hspace{5pt} unique\hspace{5pt} words}{\sqrt{2*total words}} } $
\end{center}
\vspace{5pt}

MTLD was used as a complementary measure of lexical diversity that is robust against text lengths. It calculates the
average length of word strings that maintain a minimum type-token ratio threshold, commonly .72 (this threshold was
also used in this study). A higher
MTLD
score indicates a more lexical diversity.

MTLD was implemented using a modified version of a
publicly
available
Python script \citep{MTLD_repo}. The script was adapted to support both surface-level and lemma-based calculation
modes and was integrated with the GiNZA tokenizer to handle Japanese texts. Punctuation was removed based on
spaCy's POS tags before
calculating the MTLD score. Because the measure is only sensitive to variation after a minimum number of tokens \citep{McCarthy2010},
texts with fewer than 50 tokens were excluded from MTLD calculation.

POS density measures were calculated by dividing the number of nouns, verbs(including auxiliaries), adjectives and
adverbs by the total
number of tokens in a given text. These measures provide insight into the functional distribution of lexical items and
descriptive elaboration used by learners.

%%%Formula of

\subsection{Text External Measures}
% External
%LFP - Using JLPT word lists BCCWJ
To assess lexical sophistication, the Lexical Frequency Profile (LFP) was implemented using two frequency based
resources:
\begin{itemize}
    \item The Balanced Corpus of Contemporary Written Japanese (BCCWJ) frequency list \citep{maekawa2014}, which
    provides frequency information for words appearing across a wide range of genres including books, magazine,
    blogs,
    internet
    forums, and textbooks.
    \item JLPT word lists, sourced from \citep{jisho.org}, classifies vocabulary into five levels from N5(
    basic) to N1 (advanced) based on the Japanese Language Proficiency Test (JLPT).
    \end{itemize}

In the LFP, each word in a learner text is categorized into a frequency band or JLPT level. The percentage of tokens
from each band is then calculated to assess the extent to which learners rely on high-frequency (common) versus
lower-frequency (uncommon) vocabulary  from
each of the bands.

To esure compatibility betwee the learner texts and the word lists, preprocessing was required. Word from the
frequency lists were tokenized using the same pipeline (GiNZA) as the learner texts to allow consistent matching.
 For example, some expressions such as
「もう一度」 (
\textit{mouichido}, "one more time") would appear as single entries in the vocabulary list but are tokenized
into separate
items(e.g. 「もう」and 「一度」) . By applying the same tokenizer to both the learner data and the frequency resources, this
mismatch was minimized.


%%%%%%%%%%%%%%%%%%%%%%%%%%%%%%%%%%%%%%%%
\subsection{Morphological Complexity}
%%%%%%%%%%%%%%%%%%%%%%%%%%%%%%%%%%%%%%%%
Morphological Complexity can be measured through two domains variation, which reflects the diversity of morphological
forms employed and
elaboration, which captures the internal complexity of those forms. Japanese, as an
agglutinative
language, with extensive verb inflection, auxiliary chaining, politeness marking, and compounding,
offers
numerous opportunities for both morphological variation and elaboration. To quantitatively evaluate these aspects, two
main
approaches were employed, the Morphological complexity index as proposed by
\citet{Brezina2019} and a Japanese adaptation of the Korean Readability Morphological Analyzer(KRMA)\citet{Hwang2024}. which will be
referred to as
JRMA hereafter.


\subsection{Japanese Readability Morphological Analyzer (JRMA)}

The Japanese Readability Morphological Analyzer (JRMA), is based on the KRMA developed for Korean, another
agglulative language. The JRMA indentifies and counts morphological constructions that reflect
functional and grammatical elaboration in Japanese, including verb inflections, honorific and polite forms, and noun
formations.
Morphemes
are categorized into three distinct types for analysis: Content morphemes (e.g. nouns, verbs, adjectives ),
function
morphemes (
auxilaries, and particles), and all morphemes. For each of these categories two variation measures are
calculated: the
Measure of Textual Lexical Diversity (MTLD) and the Moving Average Type-Token Ratio (MATTR).

MATTR is calculated by sliding a fixed-size window (this study used a size of 50) across the text and computing the
type-token ratio (
TTR) with in each window. The final MATTR score represents the average of all windowed TTR values.

\begin{equation}
    \text{MATTR} = \frac{1}{N} \sum_{i=1}^{N} \frac{\text{unique types}_i}{w}
\end{equation}

Here' N is the total number of windows, unique types, refers to the number of unique types in window \textit{i}, and
\textit{w} is
the
fixed window size.

MTLD is the same diversity measure as described in the lexical complexity section. However, when applied within
JRMA, it is computed separately for each morpheme category(content, function, all). Texts that didn't meet the
minimum threshold of 50 tokens were excluded from the MTLD calculation to ensure validity.

In addition to variation, JRMA includes a custom elaboration based measure, auxiliary chain density. This measure identifies auxilary-like elements that attach to main verbs, such as
auxiliary verbs (e.g.ます, ない), non-independent verbs(非自立動詞)(e.g. みる,おく), or conjunctive particles like 「て」and 「で. The
auxiliary chain density is and
computed as the
average the
number of these elements per main verb. This measure demonstrates how learners layer grammatical functions providing insight into their ability to produce structurally complex constructions.

\subsection{Morphological Complexity Index (MCI)}

Two forms of the Morphological Complexity Index (MCI) were implemented:
surface
MCI and
inflectional MCI. Surface MCI measures the diversity of the word forms at the token level, capturing variation
across all visible forms including conjugated verbs and inflected nouns. Inflectional MCI, on the other hand,
calculates the diversity of
the inflectional lemmas.

Calculation of the MCI is done by diving the text into consecutive segments of a fixed size,\textit{k}, (usually 5
or 10
tokens), and for each segment, a type token ratio (TTR) is calculated based on the number of unique surface forms in
that window. The final MCI score is the average of these TTRs across all segments.

\begin{equation}
    \text{MCI} = \frac{1}{N}\sum_{i=1}^{N} \frac{\text{Unique forms}_i}{k}
\end{equation}

Here, \textit{N} is the total number of segments, Unique Forms is the number of unique forms (surface or
inflectional) in segment \textit{i}, and \textit{k} is the fixed segment size.

As the MCI relies on sampling texts without at least 10 verbs to sample from will result on the measure being
unrepresentative due to overlap. These shorter texts were left out of the analysis for this measure.
%%%%%%%%%%%%%%%%%%%%%%%%%%%%%%%%%%%%%%%%
\section{Criterial Features}
%%%%%%%%%%%%%%%%%%%%%%%%%%%%%%%%%%%%%%%%
The selection of Criterial Features was guided by unofficial, yet widely recognized, pedagogical resources that map
grammatical structures to the Japanese Language Proficiency Test (JLPT) levels. While no offical, public documents
outline specific grammar points for each JLPT level exist, numerous reputable unofficial lists (e.g. Jisho.org \citep{jisho.org}, and
various published textbooks which were used to cross-reference the list) are available and
commonly used by learners and educators. From these resources, a comprehensive set of 154 grammar forms was
identified as potentially criterial for Japanese L2 proficiency. A complete list of all 154 grammar forms extracted
for analysis is provided in the Appendix in Table \ref{tab:Criterial-Features}.

When selecting features, priority was given to grammatical forms that are primarily form-based and exhibit
relatively consistent surface patterns, making them more amenable to automated identification and reducing
ambiguity in the matching process. This focus allowed for a robust, rule-based approach to feature extraction. While
forms across all five levels of the JLPT were included the focus of the annotation was for patterns in the
intermediate and upper levels (N3, N2, N1). This decision was based on preliminary observation from \citet{akef2025}
that the presence or absence of very basic forms (N5, N4), or very complex forms (N1), often does not sufficiently
differentiate between higher
proficiency levels, whereas the accurate use of advanced structures provides a clearer sign of development. The
distribution of these annotated forms across JLPT levels is detailed in Table
\ref{tab:CF-Counts}.

%%%%add chart of Criterial Features Breakdown%%%%%%%
\begin{table}[h!]
\centering
\begin{tabular}{cc}
\hline \textbf{JLPT Proficiency Level} & \textbf{\# of Annotated Forms} \\ \hline
N5 & 9 \\
N4  & 35 \\
N3 & 47\\
N2 & 46 \\
N1 & 17 \\
\hline
\end{tabular}
\caption{Distribution of annotated grammar forms across JLPT levels. }
\label{tab:CF-Counts}
\end{table}


A key aspect of this selection process involved identifying grammar forms that are introduced at one level but continue
to be used with more complex functions or collocations at higher levels. For instance, The particle 「しか」(shika) is
typically
introduced at N4 where it commonly appears with nouns to convey "only" or "nothing but" (e.g., 「学生しかいない。」
\textit{There are only students}). However, at N3, its usage often expands to combine other negative verb forms,
indicating a more advanced grammatical pattern(e.g., 「日本語しか話しません。」 \textit{I only speak Japanese}). This
change above in
useage is
clearly
reflected in the
surface
form, allowing for distinct rule-based identification.

However, distinguishing between different uses solely based on surface form is not always possible. Consider the
phrase「ことがる」(koto ga aru),
which
first
appears at the N4 level to state a peast experience (e.g. "have done/experienced something before"). At N3 and upper
levels, its usage expands to describe hypothetical or emphatic situations, which are highly nuanced, yet the surface
expression remains unchanged, as illustrated below.

\begin{quote}
\textbf{Example 1 (Past Experience - N4)}\\
寿司は食べた\textbf{ことがあります}。\\
\textit{'I have eaten sushi before.'}\\ 
\textbf{Example 2 (Hypothetical/Emphatic - N3+)}\\
誰でも一度泣いた\textbf{ことがある}はずだ\\
\textit{'Everyone must have cried at least once (in their lives)'}\\
\end{quote}
In such cases, where the grammatical function or nuance shifts without a corresponding change in surface morphology
or clear syntactic markers, a purely rule-based matching system, reliant on the surface annotations, would be unable to
differentiate between these distinct uses. This limitation was acknowledged during the
selection of criterial features, prioritizing forms where surface form reliably indicates the target grammar point.

\subsubsection{Rule-Based Feature Matching Function}

To automatically identify and count instances of these 154 criterial features within the preprocessed learner
corpus, a custom rule-based feature matching function was developed. This function leverages the linguistic
annotations generated by Spacy's Ginza package (tokenization, POS tagging, lemmatization, and dependency parsing) to
define specific patterns for each target grammar form.

The matching function operates by iterating through each sentence in a given writing sample and applying a set of
predefined rules. Each rule is essentially a sequence of conditions based on a token's lemma, POS tag, morphological
features, and dependency relations within the sentence. These rules are implemented as spaCy Matcher patterns,
which allow for flexible and powerful pattern matching based on token attributes.

Below is an illustrative example of a spaCy Matcher pattern designed to capture passive verb forms, typically
introduced around JLPT N4. This pattern looks for a verb followed by a specific auxiliary verb lemma ('られる' for
ichidan verbs or 'れる' for godan verbs), leveraging the POS tags 'VERB' and 'AUX' assigned by Ginza.

\begin{enumerate}
\item \textbf{Target Feature:} Passive form
\item \textbf{JLPT Level:} N4
\item \textbf{Example Pattern:}
\item \begin{lslisting}
[\# 一段 ーられる\\
            {"pos":"VERB"},\\
            {"pos": "AUX", "lemma": "られる"},\\
        ],[\# 五段 ーれる\\
            {"pos": "VERB"},\\
            {"pos": "AUX","lemma": "れる"},\\
        ]\\
\end{lslisting}
\end{enumerate}

Each of the 154 criterial features was translated into one or more such rule patterns, which were then
systematically applied across the corpus. The raw counts of these criterial
features were extracted from the text and subsequently normalized to account for variations in
essay length. These normalized frequencies were then used for further statistical analyses to
examine the
distribution and usage patterns of the criterial features across proficiency groups and to identify developmental
patterns.
%%% (Total Frequency / Total tokens)


%%%%%%%%%%%%%%%%%%%%%%%%%%%%%%%%%%%%%%%%
\section{EBMs}
%%%%%%%%%%%%%%%%%%%%%%%%%%%%%%%%%%%%%%%%
This study employed Explainable Boosting Machines (EBMs) \citep{nori2019} as the primary machine learning model for
classifying learner proficiency levels. EBMs are especially well-suited for this study as they can capture complex
nonlinear relationships between linguistic features and learner proficiency in a transparent and interpretable way.

Unlike linear models such as logistic regression, EBMs do not assume a constant linear relationship
between the features and target variable. As features within language development
sometimes involve sharp increases, plateaus, or even decreases in usage, patterns consistent with dynamic
systems
theory \citet{
Debot2007},language
development can be nonlinear and dynamic. Decision trees can model nonlinear patterns, but are often unstable and
prone to
overfitting, especially with imbalanced
datasets. EBMs strike a balance of being able to flexibly model nonlinear patterns while maintaining interpretable
outputs in the form of feature-wise plots that show how each variable affects the prediction.

However, EBMs do have limitations. One is that overlap between adjacent classes can contribute to misclassification (
especially
between N2 and N1 levels) even when using methods to mitigate class imbalance. Another is that EBMs assume each
feature is contributes independently, which may overlook important interactions between some linguistic features.
Despite
these limitations, the primary aim of this study is not to achieve perfect
classification, but to identify meaningful and interpretable relationships between linguistic
features and
proficiency.

The models were trained using the interpretml package\footnote{https://github.com/interpretml/interpret}, which
provides glass-box machine learning tools designed for interpretability. The EBM outputs include plots that make it
easy to observe how each variable contributes to the model's predictions across the JLPT proficiency scale.


\subsubsection{Data Preparation}
Prior to training the model several preprocessing steps were performed. First, removal of identifying and non-essential
features from the data, such as location, age, and participant number. JCAT scores were also removed as they
directly correlate with the JLPT scores and could bias the model.

Variables such as L1, location, gender, were initially included for
exploratory
testing, but were removed in the final model. These metadata variables  tended to dominate the model's
predictions, likely due to imbalances in the corpus (e.g. a dominant L1 group), and were subsequently removed to avoid
masking linguistic features of interest.

While native speaker texts can be used as a benchmark in language research, their inclusion is not appropriate in
the current context, which focuses on modeling the procession of L2 development. Native speakers do not form part of
the proficiency continuum represented by JLPT levels, and their linguistic profiles differ from those
of language learners. Moreover, in preliminary analysis, NS texts were found to not differ significantly from
tests from higher jLPT levels (
N1-N2).  Therefore did
not offer meaningful insight into developmental patterns across the full range of learner proficiency. For these
reasons, the native speaker group (NS) was excluded to maintain a consistent developmental frame of reference.

\subsection{Model Training and Evaluation}

To address class imbalance stratified 5-fold Cross-Validation was used to maintain proportional distribution of JLPT
levels across each fold. Random Over-sampling was also applied within each training fold to compensate for
imbalanced class sizes.

Only main effects were modeled (no pairwise interactions), to prioritize interpretability and avoid overfitting
due to the relatively small quantity of writings for the higher (N1) and lower(N5) proficiency levels.

Model performance was evaluated by aggregating predictions across folds. A confusion
matrix was used to visualize misclassifications,
across JLPT levels, and per-level breakdown of accuracy, precision, recall, and F1 scores calculated for each JLPT
level.

After cross validation, a final EBM was trained on the entire dataset to allow for global interpretation of feature
effects. This allowed for visualization of the
the most influential features and how they varied across
the JLPT proficiency levels.




