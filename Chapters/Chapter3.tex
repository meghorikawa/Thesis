\chapter{Methodology}
define  features that I will use for classification how I decided on this and motivation behind these features.
Also how I will train a model for classification 


\section{About corpus}
information about the corpus organization, prepossessing

\subsection{J-CAT test}
background and information on the J-CAT test

\section{Complexity Measures}
detail the complexity measures I will use, how I developed the scripts to automatically "extract them" and the statistical significance between the proficiency levels
list, Sent Length, Clauses per sentence, Noun phrase length,  Subordination, coordination, noun phrase length?, MTLD

\section{Design of the Classification Models}
Which Architecture?

One system just for Criterial features, One for Complexity features, and One for both. 

Random Forest or Explainable boosting machine


% Give up on this part

%\section{Criterial Features}

%\section{Automatic Error Annotation}

%Explain how I will do this so that I will further be able to extract criterial features to examine
% - Automated Error Detection?
%    can I apply an LLM to check each sentence, and then use distance measure to classify errors
%    -what about discourse errors? I have a feeling these will not be caught by an LLM...
%        disregard discourse errors.
%    need some way to measure errors when the corpus is uncorrected.

%    measure correct or incorrect particles?
%    can I train a classifier for automatic error annotation for particles?
%        using SNOW-NLP/snow_simplified_japanese_corpus for training?
%        格助詞 - が, の, を, に, へ, と , で, から, より
%        係助詞 - は, も, こぞ, でも, しか, さえ, だに
% - Which type of errors will I annotate for?
%        - Automated detection of particle errors - using work?