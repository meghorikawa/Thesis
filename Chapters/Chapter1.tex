\chapter{Introduction}

%Motivation for what I am trying to do....Language proficiency classification and which measures are reliably able to classify proficiency - using linguistic complexity measures.

%Train classification models using complexity features.... which features perform
%better?

% Why does this matter?
% - addressing a gap in research for Japanese
% - Evaluating language development/proficentcy levels in SLA
% - can automatically evaluate a lerner's proficency level

%Define problem:
% Many measures do the measures vary across languages?
%Focus of indo-european languages and English, not much info on Japanese
% unclear methods of measure


% can help inform instructional design of Japanese material?
% automated proficentcy testing?

% complexity studies have differed in their findings....and definitions of what complexity is have also been varried
% and vague
% developmental tradjectory of L2 Japanese - using traditional complexity measures and also sophistication measures
% (frequency of certain forms)

% Criterial Features:
% which features are used by lerners at each level?


Language learners' proficiency in their target language is typically assessed using established frameworks such as
the \textbf{Common European Framework of Reference for Languages (CEFR)}, and the
\textbf{American Council on the Teaching of Foreign Languages (ACTFL)}. These frameworks define proficiency levels
using generalized "can-do" statements, which help standardize assessment across multiple languages. However,
because they are designed for broad
applicability, these frameworks may
not fully capture language-specific developmental patterns, particularly in languages which significantly differ
from Indo-European languages, both in structure and in the kinds of
communicative functions commonly emphasized in early learning

In second language acquisition (SLA) research, proficiency is often analyzed in terms of three core
domains:complexity, fluency, and accuracy \cite{Skehan1989}. Of these, complexity has been the subject of extensive
investigation, particularly in written languages, where it serves as an indicator of linguistic development
\cite{Lu2010, Lu2011, Ortega2003, Iwashita2006}. Complexity measures, including syntactic and lexical features have
been widely used to evaluate proficiency, yet findings across studies have varied significantly. This inconsistency
has been attributed to differences in how complexity is defined, how it is measured, or how the tasks influence
outcomes \cite{Butle2012,Alexpoulou2017}. Although many linguistic complexity measures have been proposed, their
effectiveness in predicting proficiency remains inconsistent, particularly in L2 Japanese research, where empirical
evidence is still limited.

To address this gap, I also incorporate criterial features as a complementary but distinct class of indicators.
These are
linguistic forms that are
particularly indicative of specific proficiency levels, following
\cite{Hawkins_Buttery_2010}. Unlike general complexity measures, criterial features are grounded in developmental
benchmarks and are often reflected in illustrative proficiency descriptors. These
features help operationalize what learners can do at each stage by linking
specific linguistic structures to stages of development on a proficiency continuum.

While much of this research has focused on English and Indo-European languages, fewer studies have explored how
complexity develops in Japanese as a second language (JSL). Given that Japanese differs typologically from
Indo-European languages in aspects such as morphology, syntax, and even orthography, it is unclear whether
complexity measures and criterial features that work well for English apply equally to Japanese. Examining linguistic
complexity and useage of criterial features in L2
Japanese learners can provide new insights into second language development and help refine assessment methods.

Moreover, automated scoring and evaluation systems increasingly rely on linguistic complexity measures and
criterial features to classify
proficiency levels. However, without a clear understanding of which features best differentiate between
proficiency levels in Japanese, automated systems may lack accuracy and interpretability. Addressing this gap can
contribute to more effective automated proficiency assessment and inform the design of instructional materials
tailored towards learners of Japanese.

This thesis investigates the development of linguistic complexity in L2 Japanese learners. My analysis will
encompass traditional syntactic, lexical, and morphological complexity measures, alongside sophistication measures
such as
frequency-based indicators of linguistic structures(criterial features). This analysis will utilize corpus-based
frequency distributions
and syntactic patterns aligned with the Japanese Language Proficiency Test (JLPT) proficiency levels.
Understanding these developmental patterns can provide a more nuanced
perspective on SLA trajectories in Japanese and contribute to the development of automated proficiency classification
models. This
thesis will address the following key questions:
\begin{enumerate}
    \item Which linguistic complexity measures, and criterial features reliably indicate language development in L2
    Japanese Learners?
    \item How do linguistic complexity features and criterial features develop across different
    proficiency
    levels in Japanese
    learners?
    \item Can linguistic complexity features and/or criterial features can reliably predict proficiency
    levels in L2 Japanese
    learners?
\end{enumerate}

This thesis is structured as follows: \textbf{Chapter 2} provides background on linguistic complexity, including an
overview of complexity measures and criterial features, and their role in SLA research.
\textbf{Chapter 3} outlines the research methodology, detailing the learner corpus, the calculation of selected
complexity measures, and the extraction of criterial features.
 \textbf{Chapter 4} presents an evaluation of initial findings, focusing on the development of complexity features
across
proficiency levels and the emergence of criterial features.
\textbf{Chapter 5 } presents the findings of the classification model, while
\textbf{Chapter 6}
discusses the results and their implications for SLA research and directions for
future research.
