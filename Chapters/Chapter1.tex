\chapter{Introduction}

%Motivation for what I am trying to do....Language proficiency classification and which measures are reliably able to classify proficiency - using linguistic complexity measures.

%Train classification models using complexity features.... which features perform
%better?

% Why does this matter?
% - addressing a gap in research for Japanese
% - Evaluating language development/proficentcy levels in SLA
% - can automatically evaluate a lerner's proficency level

%Define problem:
% Many measures do the measures vary across languages?
%Focus of indo-european languages and English, not much info on Japanese
% unclear methods of measure


% can help inform instructional design of Japanese material?
% automated proficentcy testing?

% complexity studies have differed in their findings....and definitions of what complexity is have also been varried
% and vague
% developmental tradjectory of L2 Japanese - using traditional complexity measures and also sophistication measures
% (frequency of certain forms)

% Criterial Features:
% which features are used by lerners at each level?


Language learners' proficiency in their target language is typically assessed using established frameworks such as
the \textbf{Common European Framework of Reference for Languages (CEFR)}, and the
\textbf{American Council on the Teaching of Foreign Languages (ACTFL)}. These frameworks define proficiency levels
using generalized "can-do" statements, which help standardize assessment across multiple languages. However,
because they are designed for broad
applicability, these frameworks may
not fully capture language-specific developmental patterns, particularly in languages which significantly differ
from Indo-European languages, both in structure and in the kinds of
communicative functions commonly emphasized in early learning

In second language acquisition (SLA) research, proficiency is often analyzed in terms of three core
domains:complexity, fluency, and accuracy \cite{Skehan1989}. Of these, complexity has been the subject of extensive
investigation, particularly in written languages, where it serves as an indicator of linguistic development
\cite{Lu2010, Lu2011, Ortega2003, Iwashita2006}. Complexity measures, including syntactic and lexical features have
been widely used to evaluate proficiency, yet findings across studies have varied significantly. This inconsistency
has been attributed to differences in how complexity is defined, how it is measured, or how the tasks influence
outcomes \cite{Butle2012}. Many linguistic complexity measures have been proposed, but their effectiveness has
varied among studies. In L2 Japanese research, there is limited empirical evidence on which complexity measures best
predict proficiency.  To address this, I will draw on the concept of criterial features - language forms that are
particularly indicative of specific proficiency levels - as proposed by \cite{Hawkins_Buttery_2010}. These features,
often reflected in illustrated descriptors, help operationalize what learners can do at each level by linking
specific linguistic structures to stages of development across proficency levels.

While much of this research has focused on English and Indo-European languages, fewer studies have explored how
complexity develops in Japanese as a second language (JSL). Given that Japanese differs typologically from
Indo-European languages in aspects such as morphology, syntax, and even orthography, it is unclear whether
complexity measures and criterial features that work well for English apply equally to Japanese. Examining linguistic
complexity and useage of criterial features in L2
Japanese learners can provide new insights into second language development and help refine assessment methods.

Moreover, automated scoring and evaluation systems increasingly rely on linguistic complexity measures and
criterial features to classify
proficiency levels. However, without a clear understanding of which features best differentiate between
proficiency levels in Japanese, automated systems may lack accuracy and interpretability. Addressing this gap can
contribute to more effective automated proficiency assessment and inform the design of instructional materials
tailored towards learners of Japanese.

In this thesis I aim to  investigate how linguistic complexity develops in L2 Japanese learners by analyzing
both traditional syntactic and lexical complexity measures and sophistication measures (such as frequency-based
indicators of linguistic structures) by analyzing corpus-based frequency distributions and syntactic patterns across CEFR-aligned levels. Understanding these developmental patterns can provide a more nuanced
perspective on SLA trajectories in Japanese and contribute to automated proficiency classification models. Within
this thesis I will address the following:
\begin{enumerate}
    \item To identify which linguistic complexity measures, and criterial features show reliability as indices of
    language development in
    L2 Japanese learners.
    \item To examine how linguistic complexity features and criterial features develop across different proficiency
    levels in Japanese
    learners.
    \item To assess whether linguistic complexity features and/or Criterial Features can reliably predict proficiency
    levels in L2 Japanese
    learners.
\end{enumerate}

This thesis is structured as follows: \textbf{chapter 2} provides background on linguistic complexity, including an
overview of complexity measures and their role in SLA research.
\textbf{Chapter 3} outlines the research methodology, detailing the learner corpus, the calculation of selected
complexity measures, and the extraction of Criterial Features.
 \textbf{Chapter 4} presents an evaluation of initial findings, including the development of complexity features across proficiency levels and the emergence of criterial features.
\textbf{Chapter 5 }will present my findings and results of classification models, while \textbf{Chapter 6}
discusses my findings and their implications for SLA research and automated proficiency assessment, and directions for
future research. By combining traditional complexity measures with the identification of criterial features, this
study contributes both to theoretical understanding of L2 Japanese development and practical applications in
automated assesment and curriculumn design. % move this sentence to conclusion