\chapter{Introduction}

%Motivation for what I am trying to do....Language proficiency classification and which measures are reliably able to classify proficiency - automated scoring/grading of tests. etc. using linguistic complexity measures.

%Train classification models using complexity features.... which features perform
%better?

% Why does this matter?
% - addressing a gap in research for Japanese
% - Evaluating language development/proficentcy levels in SLA
% - can automatically evaluate a lerner's proficentcy level

%Define problem:
% Many measures -
%Focus of indo-european languages and English
% unclear methods of measure,


% can help inform instructional design of Japanese material?
% automated proficentcy testing?

% complexity studies have differed in their findings....and definitions of what complexity is have also been varried
% and vague
% developmental tradjectory of L2 Japanese - using traditional complexity measures and also sophistication measures
% (frequency of certain forms)

Language learners' proficiency in their target language is typically assesed using established frameworks such as the \textbf{Common European Framework of Reference for Languages (CEFR)}, and the \textbf{American Council on the Teaching of Foreign Languages (ACTFL)}. These frameworks define proficiency levels using generalized "can-do" statements, which help standardize assessment across multiple languages.  However, because they are designed for broad applicabiliy, these frameworks may not fully capture language-specific developmental patterns, particularly in languages which significantly differ from Indo-European languages.

In second language acquisition (SLA) research, proficiency is often analyzed in terms of three core
domains:complexity, fluency, and accuracy \cite(Skehan1989). of these, complexity has been the subject of extensive
investigation, particularly in written languages, where it serves as an indicator of linguistic development
\cite(Lu2010, Lu2011, Ortega2003, Iwashita2006). Complexity measures, including syntactic, and lexical features have
been widely used to evaluate proficency, yet findings across studies have varied significantly. This inconsistency
has been attributed to differences in how complexity is defined, how it is measured, or how the tasks influence
outcomes \cite(Butle2012).

While much of this research has focused on English and Indo-European languages, fewer studies have explored how
complexity develops in Japanese as a second language (JSL). Given that Japanese differs typologically from
Indo-European languages in aspects such as morphology, syntax, and even orthography, it is unclear whether
complexity measures that work well for English apply equally to Japanese. Examining linguistic complexity in L2
Japanese learners can provide new insights into second language development and help refine assessment methods.

Moreover, automated scoring and evaluation systems increasingly rely on linguistic complexity measures to classify
proficiency levels. However, without a clear understanding of which complextiy features best differenticate between
proficiency levels in Japanese, automated systems may lack accuracy and interpretability. Addressing this gap can
constribute to more effective automated proficiency assesment and inform the design of instructional materials
tailored towards learners of Japanese.


In this thesis I aim to complexity measures that are unique for evaluating Japanese, train two different models on
these features and evaluate their performance. Within this thesis I will address the following questions:
\begin{enumerate}
    \item Which syntactic complexity measures show reliability as indices for language development/proficiency level?
    \item How do linguistic complexity features develop across different proficiency levels in Japanese as a second language learners?
    \item What influence does a learner's L1 have on their developmental tradjectory of Japanese
    \item Can linguistic complexity features reliably predict proficiency levels in Japanese second language learners?
\end{enumerate}

This thesis is structured as follows: \textbf{chapter 2} provides background on complexity measures, \textbf{Chapter 3} outlines the research methodology including details on the learner corpus used, feature selection and design and training methods of the
classification models. \textbf{Chapter 4} presents an evaluation of initial findings of the features,
\textbf{Chapter 5 }will present my findings and results, \textbf{Chapter 5} discusses my
findings and their implications and directions for future research.