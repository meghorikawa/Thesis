\chapter{Background/Related Work}
%What do I need to know to understand the approach described in the methodology section

\section{Linguistic Complexity}
-What is complexity? How it is defined? in SLA literature?
    -
-Distiction between complexity, accuracy, and fluency (CAF Framework).
- Why complexity is a useful proxy for proficiency. (motivate what is being measured)
- Types of Complexity: Syntactic, lexical, morphological, discourse?


\subsection{Linguistic Complexity in SLA}
-Syntactic complexity and lexical measures as indices for proficiency levels of other languages. (Language Developmental Trajectory)
-Use of complexity in SLA studies, readability studies, and ICALL
-role of learner corpora in operationalizing and tracking complexity.
-research on non-English languages
-Existing studies on complexity in L2 Japanese - the the limitations


\subsection{Computational Approaches to Measuring Complexity}
- Introduce major complexity metrics:
    -Syntactic: sentence length, clause density, suboridation, coordination, noun phrases, depedency distance
    - Lexical: type-token ratio (TTR), MTLD, lexical sophistication
-Theoretical basis and interpretation of each measure and limitations
- Use of NLP tools in measuring complexity and automated tools in ICALL applications (CTAP)

\section{Criterial Features}
- Define criterial features: What are they and how they relate to CEFR levels.
- Illustrative descriptors vs. criterial features. (EGP)
- Frequency-based indicators of syntactic sophistication? \cite{Ellis2004}
    - comparison to developmental sequences?
- Examples of criterial features in English (and possible other languages) also describe this in relevance to Japanese
- automated extraction - POLKE


\subsection{Proficiency Assessment Frameworks}
- Overview of CEFR levels ( Is it really necessary to include ACFL?)
- The structure and goals of JLPT (grammar/lexicon targets at each level)
- Historical development of JLPT and inclusing of CEFER-style descriptors from 2012.
- limitations of JLPT, or critiques on standardized tests.
Not sure if this warrants it's own section. Describe proficency assesment measures CEFER, etc. also introduce JLPT
Can-do statements- forms teste


maybe also introduce some skepticism of the testing methods of the test?
Talk about previous work with EGP proficiency classification?
Introduce can-do statements at each of the levels and also summarize the equivalent test used in the corpus?


\section{Automatic Proficency Assesment}
- motivation - describe why automatino is needed
- Overview of existing systems using complexity/criterial features for proficency classification.
- Describe challenges
    - Learner errors impacting feature extraction (decreases robustness)
    -Inconsistent tagging/parsing in morphologically rich or low-resource languages
        -try to find something for Japanese to support some inconsistentices I found when tokenizing sample text
- Describe the aboves' relevance to Japanese, and to this study


