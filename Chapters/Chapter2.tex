\chapter{Background/Related Work}
\section{Linguistic Complexity}
Previous work on Syntactic complexity measures as indices for proficiency levels of other languages.
introduce the different measures and define them.
Syntactic, Lexical features

\subsection{Linguistic Complexity in SLA}
set definition of complexity - syntactic, lexical etc.
role of it in proficentcy research
other usess such as readability etc.
role of learner corpora?

Include:
Language Developmental Trajectory
Research of non-english languages
Previous research of Complexity in Japanese
Challenges of previous research - including Japanese


\subsection{Measuring Complexity}
Sentence Length, Clause Length, Clauses per sentence, NP length, Subordination, coordination, MTLD instead of CTTR...
Other lengths of measures and theories behind them.

\subsection{Computational approaches}
tecniques used in NLP for calculating the complexity measures - different systems that exist
uses/applications in ICALL systems


%\section{Criterial features} - gave up on this section due to corpus being uncorrected.
%What are Criterial Features? How are they decided/extracted?

%Types - Negative Positive

%Talk about previous work with EGP proficiency classification,
%% I realize this might not be so important, uless I want to choose  a few of the grammar points to analyze and try to extract....

%\section{Japanese Proficiency Language Test (JPLT)}
%JLPT levels have clear grammatical and lexical goals defined....and can do statements were added later in 2012 to 
%align with other worldwide standard proficiency tests/standards i.e. ACTFL, CEFR

%maybe also introduce some skepticism of the testing methods of the test?

%Introduce can-do statements at each of the levels and also summarize the equivalent test used in the corpus