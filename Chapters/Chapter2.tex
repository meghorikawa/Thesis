\chapter{Background/Related Work}
%What do I need to know to understand the approach described in the methodology section

\section{Linguistic Complexity}
What is complexity? How it is defined? Different types, differentiate it from other domains, accuracy...
motivate what is being measured


\subsection{Linguistic Complexity in SLA}
Previous work on Syntactic complexity measures as indices for proficiency levels of other languages.
Syntactic, Lexical features
role of it in proficiency research
other uses such as readability etc.
role of learner corpora?

Include:
Language Developmental Trajectory
Research of non-english languages
Previous research of Complexity in Japanese
Challenges of previous research - including Japanese


\subsection{Measuring Complexity and computational approaches}
introduce the different measures and define them. Also introduce techniques used in NLP for calculating the complexity
measures - different systems that exist
uses/applications in ICALL systems??

Syntactic:
Sentence Length, Clause Length, Clauses per sentence, NP length, Subordination, coordination, Dependency distance?
Lexical:
MTLD instead of CTTR... or both?
Other lengths of measures and theories behind them. Describe which measures are measuring what Also shortfalls of the
measures

\section{Criterial Features}

Frequency of forms as a form of syntactic sophistication ? \cite{Ellis2004}


\subsection{Measuring Proficency}
Not sure if this warrants it's own section. Describe proficency assesment measures CEFER, etc. also introduce JLPT
Can-do statements- forms teste

JLPT levels have clear grammatical and lexical goals defined....and can do statements were added later in 2012 to
align with other worldwide standard proficiency tests/standards i.e. ACTFL, CEFR

maybe also introduce some skepticism of the testing methods of the test?
Talk about previous work with EGP proficiency classification?
Introduce can-do statements at each of the levels and also summarize the equivalent test used in the corpus?


\section{Automatic Proficency Assesment}


Notes from Detmar:


You may need to say something abut the challenge of detecting certain
complexity indicators that may be difficult to identify robustly for
learner texts containing a lot of errors. But it is definitely not the
case that you need to include accuracy analyses in the thesis.

