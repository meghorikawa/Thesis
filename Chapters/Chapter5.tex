\chapter{Results and Discussion}
Which features contributed to classification at each level as per the EBM?

\section{Discussion}
Are any findings surprising? Why might this be? If not surprising why?


shortfalls/difficulties
- mention lerner language and the corpus being uncorrected so this may lead to misinterpretations by the tokenizer to classify things correctly.
    also difficulties in capturing lexical forms - many "mispellings" included on LFP OOV list.
- issue of some texts being too short for MTLD calculation

- use of only form-based grammar difficulty in correctly extracting meaning based forms
- also difficulty of not knowing which forms are considered one word expressions versus expressions that will
actually be broken down into smaller parts by the tokenizer (as noted below with 言い切る etc.)

In consistency in parsing of 〜切るto show completely finishing an action. 言い切る 売り切る are considered one verb. in the
case of 食べ(verb) 切る(非自立可能Verb) and 食べ(Verb) 切れる (AUX 非自立可能) makes it difficult to when dealing with rule based
matching to extract the form..死ぬほど and other constructions with ほどare also similar.

-Issues with clause parsing - the tokenizer by spacy is inaccurate in classyfying some coordinating and
subordinating conjunctions - many conjunctions in Japanese can be used in both contexts and spacy doesn't take the
context into consideration when labeling the conjunctions. - tried to mitigate this by using a rule based formula to
classify coordinating and subordinating clauses but it is difficult to control for each case. thereofore a more
robust system would possibly lead to better results......I need to say something more about clausal use in Japanese
for learners.


Need to mention the feature extractor follows rules for the standardized language which is widely taught to
learners. Variations due to dialects and casual written variations not included and therefore this would not be
able to extract text with these. Also chose grammar constructs which were largely form based (i.e. easier to extract
vs. use based) and ones that are likely to be used in written language. (leaving out spoken expressions as they
wouldn't likely appear much in writing. )