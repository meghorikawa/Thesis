\chapter{Evaluation} 
This chapter presents an evaluation of the initial results from the collected data, focusing on statistics relating to
developmental trajectories in order to identify potential developmental indices. These findings serve as the
foundation
for the feature selection used in training the explainable boosting machine. Features that demonstrate strong
performance here may also be identified as important by the Explainable Boosting Machine. The following questions
will be
addressed in this
section:

\begin{enumerate}
    \item Which measures show an overall developmental trajectory across JLPT levels?
    \item Which features significantly discriminate between proficiency levels, and which ones show a developmental
    trajectory across these levels?
\end{enumerate}

\section{Complexity Measures}

Many of the complexity measures demonstrated trends that correspond to increasing proficiency. However, the patterns
varied across measures. Some showed a consistent increase,while others
decreased or plateaued at the
higher levels. Notably, most measures were unable to distinguish between the N1 and native speaker (NS) groups.

Statistical analyses were conducted using ANOVA to test for significance, followed by Tukey's HSD test to identify
significant pairwise differences between adjacent proficiency levels.

\subsection{Syntactic Complexity Measures}

%Sentence Length
    %discriminates between ajacent proficiency levels.
%Clause per Sent
    %*Statistically significant difference between all adjacent levels except N1 and NS

Many of the syntactic measures failed to show significant differences between N1 and NS groups. This, however, is not
cause for concern as the native speaker group is not part of the JLPT scale and is used as a benchmark to
compare how close to native-like speech the other learners are. Therefore when it is mentioned that a measure
distinguished all adjacent levels, this means all levels within the JLPT proficiency scale not including the native
speaker group.
Several measures
such as sentence length, clauses per sentence, and coordinating clauses per sentence, exhibited statistical significance
across most
adjacent proficiency levels. These patterns are illustrated in Figures~\ref{fig:sentLen}, \ref{fig:cpersent}, and \ref{fig:CCperSent} respectively.

\begin{figure}[htbp]
    \centering
    \begin{minipage}{.48\textwidth}
        \centering
    \includegraphics[scale=.3]{img/sentence_len}
    \caption[The average Sentence Length across JLPT levels]{The average Sentence Length across JLPT levels}
        \label{fig:sentLen}
    \end{minipage}
    \hfill
\begin{minipage}{.48\textwidth}
        \centering
        \includegraphics[scale=.3]{img/clausesSent}
        \caption[The ratio of clauses per sentence]{Plot showing the ratio of clauses per sentence}
\label{fig:cpersent}
\end{minipage}
    \end{figure}


\subsubsection{Length-Based Measures}

Length-based metrics capture the amount of syntactic material produced per unit (e.g. per sentence, per clause,
etc.) and are surface-level indications of syntactic complexity. Sentence length exhibited a general upward trend
across proficiency levels and was statistically significant at distinguishing across most adjacent levels. This aligns
with
the idea that more advanced learners integrate more elements into a single sentence.

Noun phrase length and verb phrase length, in contrast, were less informative. NP length remained relatively stable
and did not significantly differ across levels. Verb phrase length did show a weak upward trend but lacked
statistical significance to distinguish between the proficiency levels.

\subsubsection{Clause-Based Measures (Including Coordination/Subordination)}
Clause-based measures focus on how learners structure and link clauses, capturing syntactic elaboration through
subordination and coordination.
%Clause Length
%    *N1 and N3 statistically significant difference observed
%    *slightly increases across levels.
%    *Could be influenced by tasks?. Try Removing SW1 and SW2 tasks

Clause length showed a slight
increase across proficiency levels, with a statistically significant difference in distinguishing between N1 and N3.
However, when the story-telling tasks, SW1 and SW2
tasks were removed, this measure distinguished between all adjacent upper proficiency levels, suggesting a
possible
task effect. Increases in clause length were more pronounced at the advanced stages than at the beginner levels,
hinting at its potential in detecting finer gradations of advanced proficiency.

%add other citations
Across prior studies on L2 syntactic development, a decrease in coordination and an increase in
subordination with proficiency\citep{Vyatkina2012, Lu2010,Lu2011} is commonly observed. However, in the present data,
this trend was not
entirely replicated. One reason may be the early introduction of subordinating conjunctions in Japanese L2
instruction such as 「から」(kara, \textit{because}) and 「けど」(kedo, \textit{but}), which are frequently used without
requiring transformation of verbs or deep syntactic embedding.

%Subordinate Clauses per Sent
    %statistical significance difference between N5 and N4 ,and then N1 and N3 but does not discriminate between the
%higher proficiency
    %levels. Subordination is taught quite early to L2 speakers, could this be why?
%Subordinate Clause per clause
    %statistically significant difference between N5 and N4 and N4 and N3,
    %slight decrease across levels showing a prefernce for different clause types in the higher levels?

The measure of subordinate clauses per sentence showed a modest increase across proficiency levels and was
statistically significant in distinguishing lower levels (N5 vs. N4) and non-adjacent upper levels (N3 vs N1). This
reflects developmental progression (see Figure~\ref{fig:SubclperS}).

In contrast, the ratio of subordinate clauses to total clauses showed a decreasing trend, with significant
differences observed between non-adjacent levels (N5 vs. N3, N4 vs. N2).  As shown in Figure~\ref{fig:SCperC}, this
suggests that learners at higher proficiency levels may diversify their clause structures and shift toward more
coordination or other complex constructions.

\begin{figure}[htbp]
    \centering
    \begin{minipage}{.48\textwidth}
        \centering
    \includegraphics[scale=.3]{img/SCperS}
    \caption[Average subordinate clauses to sentences across JLPT levels]{Average subordinate clauses to sentences across JLPT levels}
        \label{fig:SubclperS}
    \end{minipage}
    \hfill
\begin{minipage}{.48\textwidth}
        \centering
        \includegraphics[scale=.3]{img/SCperC}
        \caption[Average subordinate clauses to clauses ratio across JLPT levels]{Average subordinate clauses to clauses ratio across JLPT levels}
\label{fig:SCperC}
\end{minipage}
    \end{figure}

%Coordinating Conjunction per Sentence
    %*Statistically significant in distinguishing all adjacent levels except N1 and NS group
    %*Also increases across prof levels, could be due to the fact that SC are taught first.
%Coordinating Conjunction per Clause
    %*increase across prof. levels
    %*Statistically significant between levels except N2 & N1 and N1 & NS
%Subordinating Clauses to Coordinate Clauses
%    *statistically significant for every other proficiency level. ie. N5 & N3, N3 & N1, N2&N4
%    *variance also increases at the higher proficiency levels but , maybe due to more CC being used?

Measures related to coordination showed strong developmental patterns. The number of coordinate clauses per sentence
increased across levels and significantly
distinguished all adjacent levels. This trend may reflect increased syntactic range at higher levels where
coordination is used not just structurally but also stylistically (Figure~\ref{fig:CCperSent}).

Similarly, coordinate clauses per clause also increased with proficiency and distinguished most adjacent levels
except N2-N1 (Figure~\ref{fig:CCperCl}). These findings contrast with typical L2 development models, possibly due
to the instructional sequence in Japanese, where simple subordinate forms are taught earlier and coordination
becomes more common as learners diversify their expression.

The ratio of subordinate clauses to
coordinate clauses further supports this shift. It showed statistically significant differences between non-adjacent
levels (N5 vs. N3, N3 vs. N1) and increased variance at higher proficiency levels, possibly reflecting more
individualized or task-driven distribution of clause types.

\begin{figure}[htbp]
    \centering
    \begin{minipage}{.48\textwidth}
        \centering
    \includegraphics[scale=.3]{img/CCperSent}
    \caption[Average coordinate clauses to sentences across JLPT levels]{Average coordinate clauses to sentences across JLPT levels}
        \label{fig:CCperSent}
    \end{minipage}
    \hfill
\begin{minipage}{.48\textwidth}
        \centering
        \includegraphics[scale=.3]{img/CCperC}
        \caption[Average coordinate clauses to clauses ratio across JLPT levels]{Average coordinate clauses to clauses ratio across JLPT levels}
\label{fig:CCperCl}
\end{minipage}
    \end{figure}

%CC Freq
    %no significant difference between any of the levels and no clear pattern...
%SC Frequency
    %*increases across levels
    %*Significant between n5&N4, N4&N3, N3&N2, N2 & NS
Normalized frequencies of subordinating and coordinating conjunctions were analyzed as a proxy measure, as done in \citet{Vyatkina2012}, however, they showed mixed results. Subordinating conjunction frequency increased across proficiency levels and showed statistically significant differences at lower levels (N5 vs. N4, N4 vs N3), reflecting early acquisition and frequent use. Coordinating conjunction frequency, however, did not follow a clear developmental trend. This may be due to extraction limitations or the relatively sparse distribution of explicit coordinators in learner texts.

\subsubsection{Dependency-Based Measures}

Dependency-based measures provide a structural perspective on syntactic complexity by quantifying how syntactic
units are hierarchically and linearly related. Unlike length or clause-based measures, which capture surface-level
elaboration, these metrics reflect the cognitive load and integration required to process or produce a sentence.
Both measures considered, Mean Dependency Distance (MDD) and Mean Hierarchical Distance (MHD), showed strong
associations with increasing proficiency and can be considered promising indicators of developmental progression.

%MDD
    %* Statistically significant between every other level as well as the higher proficiency levels N3&N2 and N2&N1
    %*no significant between N1 and NS
    %*Increase across levels
As shown in Figure~\ref{fig:mdd}, MDD increased consistently across proficiency levels and was statistically
significant between most levels, including higher-level distinctions such as N3 vs. N2 and N2 vs. N1. The consistent
upward trajectory suggests that as learners become more proficient, they are able to manage longer dependency
relations, producing more integrated syntactic structures.

%MHD
%*Increase across levels
%*statistically significant between all adjacent levels except NS and N1

Similarly, MHD, depicted in Figure~\ref{fig:mhd}, also exhibited a steady increase with proficiency. It was
statistically significant across all adjacent levels. This measure captures the hierarchical depth of sentence
structure, and its strong performance across levels supports the idea that syntactic embedding becomes more complex
and frequent as learners advance.

\begin{figure}[htbp]
    \centering
    \begin{minipage}{.48\textwidth}
        \centering
    \includegraphics[scale=.4]{img/MDD}
    \caption[Mean Dependency Distance across JLPT Proficiency Levels]{Mean Dependency Distance (MDD) increases steadily across JLPT levels, with significant differences between most levels.}
        \label{fig:mdd}
    \end{minipage}
    \hfill
\begin{minipage}{.48\textwidth}
        \centering
        \includegraphics[scale=.4]{img/MHD}
        \caption[Mean Hierarchical Distance across JLPT Proficiency Levels]{Mean Hierarchical Distance (MHD) increases consistently, distinguishing all adjacent proficiency levels.}
\label{fig:mhd}
\end{minipage}
    \end{figure}


\subsection{Lexical Complexity Measures}
Lexical complexity captures the diversity, sophistication, and density of vocabulary used by learners. The measures
included to evaluate how vocabulary use changes across proficiency levels include type-based diversity metrics, parts
-of-speech (POS) density ratios, and lexical frequency profiles.

\subsubsection{Lexical Diversity Measures}
%CTTR
%    *Discriminates between adjacent levels at the lower proficiency levels,but no statistical significant difference
%    found between N2 & N1 and N1 & NS, and N2 and NS
%    *Generally increases between levels

The Corrected Type-Token Ratio (CTTR) showed a general increase across proficiency levels but was most effective at
distinguishing between adjacent levels in the lower proficiency range (Figure~\ref{fig:cttr}). No statistically significant
differences were
found between N2 and N1. This suggests that CTTR may plateau at advanced levels and is more sensitive to early
vocabulary expansion.

\begin{figure}[h]
    \centering
    \includegraphics[scale=.5]{img/CTTR}
    \caption[Corrected Type-Token Ratio (CTTR) across JLPT Proficiency Levels]{Corrected Type-Token Ratio (CTTR) across JLPT Proficiency Levels}
    \label{fig:cttr}
\end{figure}

%MTLD-Surface
    %* increases across levels
    %* Statistically significant in distinguishing between lower prof levels.('N2', 'N3'), ('N2', 'N4'), ('N3', 'N5')
%, ('N4','N5')
%MTLD-Inflection
    %*increases across levels not as much as with surface forms
    %* Statistically significant in distinguishing between lower prof levels.('N2', 'N3'), ('N2', 'N4'), ('N3', 'N5')
%, ('N4','N5')

The Measure of Textual Lexical Diversity was calculated for surface forms and lemmas. The
MTLD-surface measure increased steadily across proficiency levels and showed statistically significant differences
for several adjacent pairs in the lower and mid-proficiency ranges (N5 vs. N4, N3 vs. N2). This suggests a gradual
increase in word variety as learners develop their language (Figure~\ref{fig:mtldS}). The MTLD-lemma measure also
showed an increasing trend, although the effect size was smaller than for surface forms (Figure~\ref{fig:mtldL}). Statistically
significant
differences were observed between similar level pairs, suggesting that while learners increase their morphological
variation, this dimension of diversity develops more slowly.


\begin{figure}[htbp]
    \centering
    \begin{minipage}{.48\textwidth}
        \centering
    \includegraphics[scale=.4]{img/MTLDsurface}
    \caption[MTLD Surface]{MTLD Surface}
        \label{fig:mtldS}
    \end{minipage}
    \hfill
\begin{minipage}{.48\textwidth}
        \centering
        \includegraphics[scale=.4]{img/MTLDlemma}
        \caption[MTLD lemma]{MTLD Lemma}
\label{fig:mtldL}
\end{minipage}
    \end{figure}

\subsubsection{Lexical Density Measures}

Part-of-speech density ratios provide insight into how learners balance content and function words in their writing.
These ratios were normalized by total token count.
%Noun Density
    %*Decreases across proficiency levels (null subjects?)
    %*distinguishes beteween lower prof levels. N4 & N5 and N3 and N4 (almost significant p = .06)
%Verb Density
    %*increase across levels
    %* Only statistically significant at distinguishing between lower levels N5 & N4 and N4&N3
Noun density decreased across proficiency levels. This may partially be explained by the use of null subjects in
Japanese, leading to fewer overt noun phrases in more advanced writing (Figure~\ref{fig:nounDen}). Statistically significant
differences were
observed between N5-N4 and N4-N3.

Verb density increased with proficiency but plateaued at the more advanced levels (Figure~\ref{fig:verbDen}) and was only
statistically
significant in distinguishing between the lower adjacent levels (N5-N4 and N4-N3). This pattern suggests that while
verb use expands early, it stabilizes in the upper ranges.

\begin{figure}[htbp]
    \centering
    \begin{minipage}{.48\textwidth}
        \centering
    \includegraphics[scale=.4]{img/NounDen}
    \caption[Noun Density Across JLPT Proficiency Levels]{Noun Density across JLPT Proficiency Levels}
        \label{fig:nounDen}
    \end{minipage}
    \hfill
\begin{minipage}{.48\textwidth}
        \centering
        \includegraphics[scale=.4]{img/VerbDen}
        \caption[Verb Density Across JLPT Proficiency Levels]{Verb Density}
\label{fig:verbDen}
\end{minipage}
    \end{figure}


%Adjective Density
    %*Increase across proficiency levels
    %*Statistically significant in distinguishing between lower levels N5&N4, N5&N3, N3&N2
    %*could also be task dependent check later
%Adverb Density
    %*increase across levels except a drop in use for N1 (due to low sample size?)
    %*Statistically significant in distinguishing between lower levels N4&N5, N3&N5 N3&N2
Adjective density showed a consistent increase across proficiency levels and was statistically significant in
distinguishing lower and intermediate pairs (N5-N4, N5-N3, N3-N2). Adverb density also increased across levels, though with a noticeable drop at N1. This may be due to sample size
limitations or task variation. Statistically significant differences were found between N5-N4, N5-N3, and N3-N2,
indicating usefulness in assessment of early-stage proficiency.

\begin{figure}[htbp]
    \centering
    \begin{minipage}{.48\textwidth}
        \centering
    \includegraphics[scale=.4]{img/AdjDen}
    \caption[Adjective Density Across JLPT Proficiency Levels]{Adjective density across JLPT proficiency levels}
        \label{fig:adjDen}
    \end{minipage}
    \hfill
\begin{minipage}{.48\textwidth}
        \centering
        \includegraphics[scale=.4]{img/AdvDen}
        \caption[Adverb Density Across JLPT Proficiency Levels]{Adverb density across JLPT proficiency levels}
\label{fig:advDen}
\end{minipage}
    \end{figure}

\subsubsection{Lexical Frequency Profile Measures}
Lexical sophistication was examined using the Lexical Frequency Profile (LFP) and calculated based on two
reference corpora: the
Balanced Corpus of
Contemporary Written Japanese (BCCWJ) \citep{maekawa2014} and a JLPT-aligned wordlist \citep{jisho.org}. This
analysis aimed to capture how learners distribute their vocabulary use across frequency bands (in the case of BCCWJ) or
across JLPT levels,
as their proficiency develops.

%LFP -BWCCJ Corpus
    %* high frequency of items in top 1,000 band across all proficiency levels
    %*N1 and Ns group had the largest raw count of vocab used across each band
    %* MANOVA revealed P value of Exact Wilks' Lambda p-value for JLPT 4.4492403847988e-133 meaning statistically
    %significant difference exists between bands, Taking a closer look at the individual bands....
        %*1k Band ->N5&N4, N2&N3, N1&NS, Percentage of vocab from 1k band decreases as
%proficiency increases.
        %especially at teh higher proficiency level.
        %*2K Band in contrast to 1k band the 2k band sees increase in percent of vocab used across proficiency levels ,
%same for the following bands as well, even if the actual frequency decreases as explained by zipfs distribution.
%No significant difference between N1 & NS group even though there is across the other bands.
%* because of low frequency in the higher bands statistical significance between levels is From 6k band onward no
%significant statistical difference observed between any of the levels, most likely due to low frequency. because of
%this bands 6 and onwards were removed from plots, etc.

%*Out of Vocabulary - decreases across proficiency levels, perhaps due to more errors in beginnger texts? statistical
%significant between N1 and NS group as well as the beginner levels, N5&N4 and N4&N3


In the BCCWJ-based analysis, words were grouped into 1,000-word frequency bands ranging from the most frequent (1k)
to the
least frequent (10k), with an additional category for out-of-vocabulary (OOV) words. As expected, across all JLPT
levels, most lexical tokens came from the 1k band (Figure~
\ref{fig:percentBands}). A multivariate analysis of variance (
MANOVA)
confirmed that significant differences existed between JLPT levels (p<.001). Closer examination revealed that the
portion of vocabulary from the 1k band decreased in usage as proficiency increased, especially at higher levels (Figure~
\ref{fig:1kband}). Significant pairwise differences were observed between N5-N4, N3-N2, and N1-NS, suggesting that
reliance on high-frequency vocabulary diminishes with proficiency, consistent with prior findings that lexical
sophistication increases as learners advance \citep{Laufer1995}.


\begin{figure}[htbp]
    \centering
    \includegraphics[scale=.4]{img/LFP/percentBands}
    \caption[A chart of the distribution of the average percent of tokens used across frequency band]{A chart of the precentage of tokens used across frequency bands}
    \label{fig:percentBands}
\end{figure}

In contrast, vocabulary from the 2k to 4k bands showed an upward trend in usage across proficiency levels,
indicating a broadening lexical range among more advanced learners. Despite these increases, the total token count
in these bands remains relatively low due to the natural distribution of word frequencies (Zipf's Law).
Nevertheless, the 2k band showed significant differences between non-adjacent levels (e.g., N5-N3 and N4-N2), and
similar patterns were observed for the 3k and 4k bands.

From the 5k band onward, usage declined steeply, and no statistically
significant differences were observed between proficiency groups. Given their low usage and minimal discriminatory
power, bands
5k through 10k were excluded from further analysis and visualization.


\begin{figure}[htbp]
    \centering
    \begin{minipage}{.48\textwidth}
        \centering
    \includegraphics[scale=.4]{img/LFP/1k}
    \caption[Percentage of tokens used from the BCCWJ Corpus 1k band]{A chart of the average percent of tokens used across the 1k frequency band of the BCCWJ Corpus}
        \label{fig:1kband}
    \end{minipage}
    \hfill
\begin{minipage}{.48\textwidth}
        \centering
        \includegraphics[scale=.4]{img/LFP/2k}
        \caption[Percentage of tokens used from the BCCWJ Corpus 2k band]{A chart of the average percent of tokens used across the 2k frequency band of the BCCWJ Corpus}
\label{fig:2kband}
\end{minipage}
    \end{figure}


\begin{figure}[htbp]
    \centering
    \begin{minipage}{.48\textwidth}
        \centering
    \includegraphics[scale=.4]{img/LFP/3k}
    \caption[Percentage of tokens used from the BCCWJ Corpus 3k band]{A chart of the average percent of tokens used across the 3k frequency band of the BCCWJ Corpus}
        \label{fig:3kband}
    \end{minipage}
    \hfill
\begin{minipage}{.48\textwidth}
        \centering
        \includegraphics[scale=.4]{img/LFP/4k}
        \caption[Percentage of tokens used from the BCCWJ Corpus 4k band]{A chart of the average percent of tokens used across the 4k frequency band of the BCCWJ Corpus}
\label{fig:4kband}
\end{minipage}
    \end{figure}

Out-of-Vocabulary (OOV) items also decreased with proficiency. Statistically significant
differences were found at the lower
 proficiency levels (N5-N4,
N4-N3), as well as distinguishing between the N1 and native speaker group. This
suggests that lower-level learners
may
have used more off-list or erroneous forms, while more proficient speakers
had more controlled lexical output.
\begin{figure}
    \centering
    \includegraphics[scale=.4]{img/LFP/OOV}
    \caption[Percentage of tokens in the Out of Vocabulary list (OOV)]{A chart of the average percentage of tokens in the out of vocabulary list from the BCCWJ Corpus, across JLPT levels}
    \label{fig:LFPOOV}
\end{figure}

%LFP - JLPT WordList
    %*percentage of N5 words slightly decreases across proficiency levels to show higher learner's preference for
    %vocabulary from the higher lists (just like for the BCCJW).
    %N5- No statistical difference between the beginner-intermediate levels (N5-N3) then N3-N2 shows statistical
%significant different (meaning this can be used to differentiate between the beginning and advanced levels)

%N4 - increases across JLPT levels. Statistical significance between N5&N4 and N4&N3

%N3 - not noteable at all the vocabulary use at this level stayed flat and nothing was statistically significant.

%N2- Significant difference between all levels but the lower N5&N4. also increases across proficiency levels.

%N1 - statistical significance between N5&N4, N4&N3, but not at distinguishing the higher proficiency levels.\

An additional analysis was conducted using JLPT-aligned vocabulary lists. Results from this alternative
approach were broadly consistent with the BCCWJ-based findings. Words from the
N5 list were commonly used across all proficiency levels, but their relative frequency declined steadily as
proficiency increased.
While no statistically significant differences were found among the beginner groups (N5-N3), a marked shift
was observed between
N3 and N2, indicating that N5 vocabulary use may help distinguish between beginner and advanced learners(
Figure~\ref{fig:JLPTN5vocab}).

N4 level vocabulary showed a gradual increase across levels peaking around N2, followed by a slight decline 
at N1 (Figure~
\ref{fig:JLPTN4vocab})
. Statistically significant
differences were observed between N5 and N4 and
between N4 and N3. In contrast, N3 vocabulary use remained relatively flat across levels with no 
significant 
differences,
indicating that these words may not effectively distinguish between proficiency groups (Figure~\ref{fig:JLPTN3vocab}).

\begin{figure}[htbp]
    \centering
    \begin{minipage}{.48\textwidth}
        \centering
    \includegraphics[scale=.4]{img/LFP/JLPT_N5}
    \caption[Percentage of tokens used from the JLPT N5 List]{The Percentage of tokens used from the JLPT N5 List}
        \label{fig:JLPTN5vocab}
    \end{minipage}
    \hfill
\begin{minipage}{.48\textwidth}
        \centering
        \includegraphics[scale=.4]{img/LFP/JLPT_N4}
        \caption[Percentage of tokens used from the JLPT N4 List]{The Percentage of tokens used from the JLPT N4 List}
\label{fig:JLPTN4vocab}
\end{minipage}
    \end{figure}
N2 vocabulary demonstrated a consistent upward trajectory and significantly differentiated all levels
except the two lowest (N5 and N4) (Figure~\ref{fig:JLPTN2vocab}). Finally, N1 increased in usage but only showed
significance in
distinguishing
the lower levels (N5-N4, N4-N3) and not between the more advanced groups (Figure~\ref{fig:JLPTN1vocab}). This
may be due to the nature
of N1 vocabulary, which often includes low-frequency or domain-specific terms that may not be easily elicited
through the given writing tasks. 

\begin{figure}[htbp]
    \centering
    \begin{minipage}{.48\textwidth}
        \centering
    \includegraphics[scale=.4]{img/LFP/JLPT_N3}
    \caption[Percentage of tokens used from the JLPT N3 List]{The Percentage of tokens used from the JLPT N3 List}
        \label{fig:JLPTN3vocab}
    \end{minipage}
    \hfill
\begin{minipage}{.48\textwidth}
        \centering
        \includegraphics[scale=.4]{img/LFP/JLPT_N2}
        \caption[Percentage of tokens used from the JLPT N2 List]{The Percentage of tokens used from the JLPT N2 List}
\label{fig:JLPTN2vocab}
\end{minipage}
    \end{figure}

\begin{figure}
           \centering
           \includegraphics[scale=.5]{img/LFP/JLPT_N1}
           \caption[Percentage of tokens used from the JLPT N1 List]{The Percentage of tokens used from the JLPT N1 List}
           \label{fig:JLPTN1vocab}
\end{figure}


\subsection{Morphological Complexity Measures}
This section reports on morphological complexity through two indices: the Morphological Complexity Index (MCI)
\citep{Brezina2019} and JMRA, based on the Korean Morphological Richness Analyzer\citep{Hwang2024}. These
measures aim to observe how learners employ morphological elaboration, either through surface variation or content
and function word use.

\subsubsection{MCI Measures}

%MCI 5-Surface
%\begin{itemize}
%   \item *33 texts removed
%   \item  *Increase across levels
%    \item *statistically significant across groups except N1&NS and N1 and N2
% \end{itemize}

The Morphological Complexity Index was computed using both surface and inflectional types across two window sizes (
MCI-5 and MCI-10). For MCI-5 using surface forms, 33 texts were excluded due to insufficient token count. A clear
increase in complexity was observed across proficiency levels, with statistically significant differences between
most groups except N2-N1 (Figure~\ref{fig:MCI5surface}). This suggests that the measure is sensitive to early and
intermediate stages of
development
but plateaus among advanced learners.


\begin{figure}[htbp]
    \centering
    \begin{minipage}{.48\textwidth}
        \centering
    \includegraphics[scale=.4]{img/MCI5surface}
    \caption[Average MCI 5 surface scores across Proficiency levels]{Average MCI 5 surface scores across proficiency levels}
        \label{fig:MCI5surface}
    \end{minipage}
    \hfill
\begin{minipage}{.48\textwidth}
        \centering
        \includegraphics[scale=.4]{img/MCI5inflection}
        \caption[Average MCI 5 inflection scores across Proficiency levels]{Average MCI 5 inflection scores across JLPT proficiency levels}
\label{fig:MCI5inflection}
\end{minipage}
    \end{figure}

%MCI 5-Inflection
%    *33 texts dropped
%    *not as great an increase across levels
%    *does not discriminate between adjacent levels but every other level at lower levels N5&N3, N4&N2

In contrast, the inflection-based MCI-5 measure showed a weaker overall trend (Figure~\ref{fig:MCI5inflection}).
While it captured
statistically
significant differences between non-adjacent and lower-level pairs, such as N5-N3 and N4-N2, it failed to distinguish
between adjacent levels, indicating lower sensitivity compared to the surface-based version.


%MCI 10 - Surface (621 texts dropped)
    %* Increase across levels
    %* Statistically significant at lower levels N5 - N2 , no significance between N2, N1, NS
%MCI 10 - Inflection
%    *statistically significant difference between N1 and NS!
%    *Statistically significant at lower levels N5-N2, no significance between N2 and N1

MCI-10 results mirror these trends but were more restrictive. Due to the larger window size, the surface-based MCI-10
required longer texts, leading to the exclusion of 621 samples due to low token count.  Despite this, it showed an
upward trend in complexity and statistically significant differences up to the N2 level (Figure~\ref{fig:MCI10surface}). However, no
differences were
observed among the highest levels, again indicating a ceiling effect. The inflection-based MCI-10 was the only
version that yielded a significant distinction between the N1 and native speaker groups, although its discriminatory
power was also limited to the lower levels otherwise (Figure~\ref{MCI10inflection}).


\begin{figure}[htbp]
    \centering
    \begin{minipage}{.48\textwidth}
        \centering
    \includegraphics[scale=.4]{img/MCI10surface}
    \caption[Average MCI 10 surface scores across Proficiency levels]{The average MCI 10 surface score across JLPT Proficiency levels}
        \label{fig:MCI10surface}
    \end{minipage}
    \hfill
\begin{minipage}{.48\textwidth}
        \centering
        \includegraphics[scale=.4]{img/MCI10inflection}
        \caption[Average MCI 10 inflection scores across Proficiency levels]{The average MCI 10 inflection score across JLPT Proficiency levels}
\label{fig:MCI10inflection}
\end{minipage}
    \end{figure}

Overall, surface-based MCI-5 emerged as the most practical and sensitive measure, especially for lower proficiency
learners who are most likely to have shorter texts. It retains more data and yields strong significance patterns
without the data-loss issues introduced by MCI-10's longer window size requirement.

\subsubsection{JMRA-Based Measures}

%*MTLD measures had dropped texts due to the minimum text length limit required. This is especially true for
%these measures as it is using specific function/content words meaning even longer text is needed.
Morphological diversity was also evaluated using JMRA-based variants of MTLD and MATTR. Due to the structural
specificity of JMRA (focusing on function vs. content words), these metrics have stricter text length requirements,
leading to large numbers of text exclusions.

\begin{figure}[htbp]
    \centering
    \begin{minipage}{.48\textwidth}
        \centering
    \includegraphics[scale=.4]{img/JMRA-MTLD-all}
    \caption[Average JMRA MTLD all score across proficiency levels]{Average JMRA MTLD all(content and function) morphemes score across proficiency levels}
        \label{fig:MTLDall}
    \end{minipage}
    \hfill
\begin{minipage}{.48\textwidth}
        \centering
        \includegraphics[scale=.4]{img/JMRA-MTLDfunction}
        \caption[Average JMRA MTLD function morphemes score across proficiency levels]{Average JMRA MTLD function morphemes score across proficiency levels.}
\label{fig:MTLDfunction}
\end{minipage}
    \end{figure}
%JMRA_all_MTLD
   % *49 rows removed
    %*statistically significant difference in distinguishing N5&N4, N3&N2 levels

%JMRA_function_MTLD
    %*2365 rows removed
    %* only difference in N4 and N3 were statistically significant.
    %*no clear pattern of increasing/decreasing
%JMRA_content_MTLD
%    * 2433 rows removed of 4840
%    * Statistically significant in distinguishing the lower levels. N4&N5 N3&N2

For the JMRA MTLD calculation on the combined list of function and content words (JMRA\_all\_MTLD), 36 rows were removed
for not
meeting the minimum token length requirement. The measure successfully distinguished N5-N4 and N3-N2 non-adjacent
levels, with a
general upward trend (Figure~\ref{fig:MTLDall}). However, the MTLD score for the function word list led to the removal
of 2,365
texts due to
text length, leading to weaker results (Figure~\ref{fig:MTLDfunction}). Consequently, only significance in
distinguishing
between N4-N3
levels was observed, and no consistent developmental pattern was observed. Content-based MTLD required even longer
texts; 2,433 texts were excluded (close to half of texts). Nevertheless, it captured meaningful differences at the
lower
levels in distinguishing between N5-N4 and N3-N2 levels (Figure~\ref{fig:MTLDcontent}). 


\begin{figure}[htbp]
    \centering
    \begin{minipage}{.48\textwidth}
        \centering
    \includegraphics[scale=.4]{img/JMRA-MTLDcontent}
    \caption[Average JMRA MTLD content morpheme score across proficiency levels]{Average JMRA MTLD content morphese score across proficiency levels}
        \label{fig:MTLDcontent}
    \end{minipage}
    \hfill
\begin{minipage}{.48\textwidth}
        \centering
        \includegraphics[scale=.4]{img/JMRA-MATTRall}
        \caption[Average JMRA MATTR all morpheme score across proficiency levels]{Average JMRA MATTR all morpheme score across proficiency levels}
\label{fig:MATTRall}
\end{minipage}
    \end{figure}



%JMRA_all_MATTR
    %*Significant difference between all levels except the N1&NS,and N1&N2
    %* slight increase across levels, variance also decreases

%JMRA_Function_MATTR
    %*Slight decrease across levels.
    %*decrease in variance across levels
    %* significant pairs: (N5,N4), (N4,N3),(N5,N3)

%JMRA_Content_MATTR
    %* decrease in variance across levels
    %* significant pairs: (N2,N3), (N5,N3)

For the MATTR-based variants, the combined list of function and content words showed a slight increase in diversity
and a reduction in variance across proficiency levels. It yielded statistically significant differences between all
levels except N1-N2 (Figure~\ref{fig:MATTRall}). 

\begin{figure}[htbp]
    \centering
    \begin{minipage}{.48\textwidth}
        \centering
    \includegraphics[scale=.4]{img/JMRA-MATTRcontent}
    \caption[MATTR content]{}
        \label{fig:MATTRcontent}
    \end{minipage}
    \hfill
\begin{minipage}{.48\textwidth}
        \centering
        \includegraphics[scale=.4]{img/JMRA-MATTRfunction}
        \caption[MATTR function]{}
\label{fig:MATTRfunction}
\end{minipage}
    \end{figure}

Function-based MATTR showed a slight decrease in diversity and
variance, distinguishing N5-N4 and N4-N3 levels (Figure~\ref{fig:MATTRfunction}). Content-based MATTR exhibited a
similar decline in variance and showed significant contrasts for N5-N3 and N3-N2 (Figure~\ref{fig:MATTRcontent}).

\begin{figure}[htpb]
\centering
\includegraphics[scale=.5]{img/auxchains}
\caption[Average length of auxiliary verb chains across JLPT Proficiency Levels]{Average length of auxiliary verb chains across JLPT Proficiency Levels}
\label{fig:auxchain}
\end{figure}
%Auxiliary Chains
    %* significant pairs (N5,N4),(N3,N5)
    %* increase from N5 to N4, but after that plateaus. Meaning that after the beginning levels chain size doesn't
    %increase
Lastly, auxiliary chain length was explored as a local morphological measure. It revealed a significant increase
from N5 to N4, but plateaued afterward (Figure~\ref{fig:auxchain}), suggesting that auxiliary chaining develops
primarily
at the
initial stages
and stabilizes among intermediate and advanced learners.


\subsection{Summary}
%Write here summarizing the most demonstrative complexity measures:
%i.e. Syntactic: Sent Length,Coordinate Clauses per sentence,  MDD, MHD
%Lexical: CTTR over MTLD? BCCJW corpus or JLPT wordlist? I'm not sure.
%Morphological: MCI5-surface, MATTR-all (combined function and content word list)
Several complexity measures stood out as particularly effective in capturing developmental trends across JLPT
proficiency levels.
From a syntactic standpoint, sentence length, coordinate clauses per sentence, and both Mean Dependency Distance (
MDD) and Mean Hierarchical Distance (MHD) consistently reflected increasing structural complexity. These measures
reliably distinguished between adjacent levels, especially in the intermediate to advanced range.

For lexical complexity, CTTR was sensitive to development at lower levels but plateaued later. While MTLD is
theoretically robust, its utility was limited due to text-length constraints. The BCCWJ-based Lexical Frequency
Profile proved more informative than the JLPT wordlist, especially in showing shifts in vocabulary distribution
across the 1k-4k bands.

In terms of morphological complexity, the MCI-5 surface form index was the most practical and sensitive measure,
avoiding the data-loss seen with MCI-10 while still capturing developmental patterns. JMRA all-MATTR measure, combining
both
function and content word diversity, also showed a clear progression across levels and offered a stable profile of
morphological variation.

Overall, sentence length, coordinate clauses per sentence, MDD/MHD, CTTR, BCCWJ-based LFP, MCI-5 surface, and JMRA
all-MATTR were the most reliable indicators of linguistic development.

\section{Criterial Features}

%look at how many of structures from each level are used across all features used, just as was observed in \citet{
%akef2025} frequencies of the individual forms was low possibly due to stylistic choice or avoidance due to
%unfamiliarity etc.

%Due to the large amount of 0 in the data a Kruskal-wallis test and then Dunn's was run to analyze significance
%between pairs.

%In terms of development, N5 learners used 「~たい」(tai, to express desire to do something) the most, N4 learners also
%increasingly used 「~たい」 then from N3 and onwards the passive form (受身形, Ukemikei) is used increasingly across
%levels.

%Ukemi use increases across levels, Tai is used more frequently in the beginning levels and use decreases in N2 and
%N1, Tekuru use also increases across levels used most frequently by the Native speaker group. Tara plateaus at N3.
%Toomou peaks at N4 and then begins to decrease in the higher proficiency levels.

%tai with kruskal willis found to be nonsignificant, other grammar points were found to be significant (p<.005)


This section examines the developmental trajectory of specific Japanese grammar forms across different proficiency
levels (JLPT N5 to Native Speaker), investigating their potential as 'criteria features' for assessing linguistic
development. As observed in previous research, (e.g., \citet{akef2025}), the absolute
frequencies of individual grammar
forms are often relatively low, possibly due to individual stylistic preferences, avoidance strategies, or incomplete
acquisition.

To account for differences in text length, all grammar forms were normalized by token count (i.e., form-to-token
ratio). Given the large number of zero values, non-parametric tests specifically Kruskal-Wallis followed by Dunn's
post hoc comparison for pairwise significance were employed. The statistical significance of these pairwise
comparisons between proficiency levels for selected grammar forms is visualized in Figure~\ref{fig:CFheatmap}, which
presents a heatmap of Dunn's test p-values.


\subsection{Overall Grammar Form Frequencies}
To provide context on the general prevalence of grammar forms in the I-JAS corpus, Table~\ref{tab:CF-form-freq}
presents the raw counts of the top 10 most frequent forms extracted from the learner texts. Complementing this,
Table~\ref{
tab:CF-form-freq-ratio} provides their normalized frequencies (ratio of form to token count), which controls for
variations in text length and offers a clearer picture of their proportional use across the entire corpus.


\begin{table}[h!]
\centering
\begin{tabular}{ccc}
\hline \textbf{Form} & \textbf{\# JLPT Level} & \textbf{Raw Count} \\ \hline
受身形 (ukemi; passive)                & N4 & 2529 \\
〜たい (tai; desire)                  & N5 & 2156 \\
〜と思う (to omou; to think)          & N4 & 1981 \\
〜という (to iu; to quote)            & N3 & 1389 \\
〜たら (tara; conditional)            & N4 & 1330 \\
〜ように (youni; so that/in order to) & N4 & 948 \\
〜てくる (tekuru; aspectual)          & N4 & 926 \\
〜させる (saseru; causative)          & N4 & 555 \\
〜てください (tekudasai; request)     & N5 & 533 \\
〜たびに (tabini; each time)          & N3 & 527 \\
\hline
\end{tabular}
\caption[Raw frequency of Top 10 Forms extracted from the I-JAS Corpus]{The raw count of the top 10 forms extracted
from the learner texts}
\label{tab:CF-form-freq}
\end{table}

\begin{table}[h!]
\centering
\begin{tabular}{ccc}
\hline \textbf{Form} & \textbf{\# JLPT Level} & \textbf{Ratio} \\ \hline
〜たい (tai; desire)                      & N5 & 0.00285 \\
受身形 (ukemi; passive)                  & N4 & 0.00270 \\
〜と思う (to omou; to think)             & N4 & 0.00201 \\
〜たら (tara; conditional)               & N4 & 0.00158 \\
〜てくる (tekuru; aspectual)             & N4 & 0.00126 \\
〜という (to iu; to quote)               & N3 & 0.00121 \\
〜ように (youni; so that/in order to)    & N4 & 0.00078 \\
〜てください (tekudasai; request)        & N5 & 0.00066 \\
〜させる (saseru; causative)             & N4 & 0.00065 \\
〜つもり (tsumori; intention)            & N5 & 0.00058 \\
\hline
\end{tabular}
\caption[Top 10 Grammar forms by normalized token ratio extracted from the I-JAS Corpus]{Normalized frequency of top
10 grammar forms. Normalization was performed using the ratio of form to token count to control for text length. }
\label{tab:CF-form-freq-ratio}
\end{table}

\subsection{Developmental Trajectories of Specific Grammar Forms Across JLPT Levels}

Five grammar forms were selected for closer analysis based on their frequency and theoretical relevance:

\begin{itemize}
\item \textbf{「〜たい」}(N5) - \textit{tai}, used to express desire.
\item \textbf{「〜たら」}(N4) - \textit{tara}, conditional clause marker.
\item \textbf{「〜てくる」}(N4) - \textit{tekuru}, aspectual or completive form.
\item \textbf{「〜とおもう」}(N5) - \textit{to omou}, expressing thoughts or uncertainty.
\item \textbf{受身形}(N5) - \textit{ukemikei}, the passive form.
\end{itemize}

Figure~\ref{fig:CFplot} presents a general line plot illustrating the usage ratio of each of these grammar forms
across the proficiency levels, allowing for an initial qualitative observation of developmental trends. The detailed
analysis of these forms, including their specific trajectories and statistical significance from the Dunn's post hoc
tests (Figure~\ref{fig:CFheatmap}), reveals distinct patterns indicative of learner development.


\begin{figure}[h]
\centering
\includegraphics[scale=.5]{img/CF-plot}
\caption[The use of the top 5 frequently used grammar forms across JLPT levels]{The use of the top 5 frequently used grammar forms across JLPT levels}
\label{fig:CFplot}
\end{figure}


Among beginner learners (N5 and N4), 「〜たい」(tai) was frequently used, as seen in Figure~\ref{fig:CFplot}, with
ratios of 0.00288 (N5) and 0.00301 (N4) as seen in Table~\ref{tab:CF-top6-per-level}.
However, its
use gradually declined
with
increasing proficiency, showing ratios of 0.00255 (N2) and 0.00240 (N1), and was not statistically significant across
groups ($p$=
0.5821),
which is confirmed by the
lack of
significant p-values in Figure~\ref{fig:CFheatmap} for any pairwise comparison involving this form. This suggests that
while used by beginners, it does not serve as a strong discriminator for developmental stages. This aligns
with previous
findings that demonstrated early acquired features plateau at specific stages and do not discriminate well across
levels \citep{Ortega2003}.

One possible
explanation for this pattern is that 「〜たい」expresses desire in a straightforward, formulaic manner and is taught very
early in most Japanese language curricula. Its functional simplicity and direct mapping to communicative intent make
it highly accessible for beginners and pragmatically useful across many task types. Interestingly, the native
speaker group also demonstrated relatively high usage of 「〜たい」, exceeding that of the advanced learners (N2 and N1)
. This form's persistence among native speakers may reflect its broad utility, especially in tasks
requiring expressions of personal intent, preferences, or future plans (such as those in the I-JAS). Thus,
rather than
signaling proficiency, perhaps the use of this form may reflect genre conventions and discourse functions encouraged
by the task.


\begin{figure}[h!]
\centering
\includegraphics[scale=.4]{img/CFheatmap}
\caption[Heatmap of top 5 grammar forms]{Heatmap of Dunn's post hoc p=values for the top 5 grammar forms across JLPT levels, illustrating pairwise statistical significance}
\label{fig:CFheatmap}
\end{figure}
In
contrast, the
passive form (受身形) showed a clear upward trend in usage across proficiency levels (Figure~\ref{fig:CFplot}),
starting at 0.00139(N5) and consistently increasing to 0.00347 (N1) (Table~\ref{tab:CF-top6-per-level}). It was statistically
significant
across many level comparisons ($p$<.001), including N5-N4, N4-N3, and N4-N2 as indicated by the significant p-values in
Figure~\ref{fig:CFheatmap}. This consistent progression demonstrates its strong potential as a criterial feature
for
tracking
development in the lower to intermediate
proficiency ranges. However, its developmental signal appeared to flatten at higher levels, as it was not significant
between certain adjacent levels(e.g., N3-N2, N2-N1). Nonetheless, the effect sizes observed between beginner
and advanced groups underscore its value in capturing major developmental differences.

「〜てくる」(tekuru), an aspectual form, demonstrated a steady increase in frequency across proficiency
levels (Figure~\ref{fig:CFplot}), with its ratio growing from 0.00098 (N4) to .00217 (N1) and reaching
its highest at 0.00301(NS) (Table~\ref{tab:CF-top6-per-level}). This form was
statistically significant across almost all JLPT levels, as shown by the significant p-values for most pairings in
Figure~\ref{CFheatmap}. Exceptions included N2-N1 and N2-N3, indicating a possible plateau or more nuanced
development at the higher intermediate to advanced stages. This sustained increase suggests that while the basic
temporal or motion-related usage of 「〜てくる」may be acquired earlier, its full functional range, including
discourse-pragmatic or aspectual nuances such as expressing emotional change, backgrounding an action or event, or
temporal layering, could require more advanced proficiency, making it a valuable marker for advanced linguistic
development.

%TODO include an analysis of the extracted forms

The usage of 「〜とおもう」 (to omou) peaked at the N4 level (0.00234), followed by a noticeable decline in the higher
proficiency
groups (0.00183 at N2, 0.00146 at NS) (Figure~\ref{fig:CFplot}, Table~\ref{tab:CF-top6-per-level}). Its
use was
significant across most pairings up to N1 as shown in Figure~\ref{fig:CFheatmap}. This pattern might reflect its
transitional
role, perhaps being a common early strategy for expressing
internal states or
opinions before learners acquire a wider range of more nuanced expressions as their proficiency advances. This can 
be understood in the context of \citet{selinker1972}'s concept of "fossilized forms," where learners may rely heavily
on certain high-frequency forms in earlier stages. As proficiency increases and the learner's interlanguage system
expands and diversifies, the reliance on such forms may decrease.

The conditional marker,「〜たら」(tara) appeared to plateau around the intermediate levels, with its ratio reaching
0.00192 (N3). Despite this, it was still statistically significant across several
non-adjacent
groups, including N5-N3 and
N4-N2,
as indicated by the significant p-values in Figure~\ref{fig:CFheatmap}. This suggests its utility in discriminating
between certain non-adjacent or adjacent lower-to-mid proficiency levels, even if its overall trajectory shows a
more stable presence at intermediate stages rather than continuous growth. The low use among the native speaker
group (NS) is also notable, potentially indicating that, native speakers prefer more advanced or alternative conditional
forms.

%Talk about this chart and how Tekudasai only appears at the beginner level N5 and that the causative form appears at
%N1. integrate into the rest of the section so it isn't redundant.....
Beyond these five primary grammar forms, Table~\ref{tab:CF-top6-per-level} also provides insights into other grammar
points across the proficiency levels. The table clearly shows that 「〜てください」(tekudasai), a common polite request
form, appears in the top 6 forms exclusively at the N5 (beginner) level with a ratio of 0.00084. This strongly
suggests it as a feature of early-stage production, likely reflecting its direct instruction and high communicative
utility for novices. Conversely, the causative form 「〜させる」(saseru) is notably absent from N5 to N2 and appears only
at the N1 and native speaker levels, with its highest ratio in the native speaker group (0.00121). This strongly
suggests it as a marker of high proficiency acquisition, reflecting the inherent syntactic and semantic complexity
of causative constructions that are typically acquired later. 「〜という」(to iu) also shows a gradual, albeit subtle,
increase in normalized ratio from N5 to N1 (0.00105 to 0.00145), indicating a steady acquisition.

\begin{table}[h!]
\centering
\small
\resizebox{\textwidth}{!}{
\begin{tabular}{>{\raggedright\arraybackslash}m{3.5cm}lccccc}
\hline
\textbf{Form} & \textbf{N5} & \textbf{N4} & \textbf{N3} & \textbf{N2} & \textbf{N1} & \textbf{NS} \\
\hline
\p{〜たい (tai)\\ \textit{desire}} & 0.00288 & 0.00301 & 0.00289 & 0.00255 & 0.00240 & 0.00305 \\
\p{〜と思う (to omou) \\ \textit{to think}} & 0.00160 & 0.00234 & 0.00210 & 0.00183 & 0.00206 & 0.00146 \\
\p{〜たら (tara)\\ conditional} & 0.00142 & 0.00135 & 0.00192 & 0.00175 & 0.00187 & --- \\
\p{受身形(ukemi)\\ passive }& 0.00139 & 0.00237 & 0.00303 & 0.00337 & 0.00347 & 0.00367 \\
\p{〜という (to iu)\\ \textit{to quote}} & 0.00105 & 0.00115 & 0.00120 & 0.00133 & 0.00145 & 0.00142 \\
\p{〜てくる (tekuru)\\ aspectual} & --- & 0.00098 & 0.00132 & 0.00137 & 0.00217 & 0.00301 \\
\p{〜てください (tekudasai)\\ polite request)} & 0.00084 & --- & --- & --- & --- & --- \\
\p{〜させる (saseru)\\ causative} & --- & --- & --- & --- & --- & 0.00121 \\
\hline
\end{tabular}}
\caption[Top 6 Grammar Forms by use Ratio across JLPT Levels]{Top 6 Grammar Forms by Ratio Across JLPT Levels (Top 6 per level)}
\label{tab:CF-top6-per-level}
\end{table}



\subsection{Summary and Implications}
Overall, while the frequency of many individual forms remains relatively low, their distinct developmental trajectories
suggest potential as
criterial features. Particularly the passive form (受身形) and 「〜てくる」(tekuru) emerge as particularly strong candidates
for tracking proficiency development due to their consistent upward trends and widespread statistical significance
across multiple proficiency level comparisons. These findings indicate that a comprehensive developmental index
could benefit significantly from incorporating measures of such forms, especially those showing clear, continuous
progression across proficiency levels. The observed plateauing or declining usage of other forms (e.g. 「〜たい」(tai),
「〜とおもう」(to omou)) also provides valuable information, suggesting these might be useful for distinguishing specific,
earlier stages of acquisition.